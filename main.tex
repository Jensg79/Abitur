\documentclass[a4paper]{report}
\usepackage[utf8]{inputenc}
\usepackage[ngerman]{babel}
\usepackage{amsmath, amsfonts, amsthm, graphicx, geometry, lipsum}
\geometry{verbose,a4paper,tmargin=25mm,bmargin=25mm,lmargin=20mm }%,rmargin=20mm}
\usepackage{hyperref}
\usepackage{xcolor}
\usepackage{polynom}
\polyset{%
    vars=x,
    style=C,
    div=:,
    delims={(}{)}}
\usepackage{dsfont}
\colorlet{shadecolor}{yellow!15}
\usepackage{thmtools}
\hypersetup{colorlinks =true, linkcolor = blue, urlcolor =red, pdftitle={Mathematik in Q11}}
\usepackage{fancyvrb}
\usepackage{fancyhdr, lastpage}
\pagestyle{fancy}
\usepackage{xcolor}

\lhead{J. Gundelwein}
\rhead{Gymnasium Gröbenzell}
\cfoot{Seite \thepage \ von \pageref{LastPage}}
\usepackage[Sonny]{fncychap}
%Options: Sonny, Glenn, Lenny, Conny, Rejne, Bjarne, Bjornstrup

\usepackage{xcolor}
\usepackage{afterpage}
\usepackage{tikz}
\usetikzlibrary{quotes, angles}
\usepackage[most]{tcolorbox}
\usepackage{pgfplots}
\pgfplotsset{compat=1.15, width =10cm}
\usepackage{mathrsfs}
\usetikzlibrary{arrows}
\newtcbtheorem[number within=section]{satz}
    {Satz}{colback=green!7,colframe=gray!85!black,fonttitle=\bfseries,title after break={Satz (Fortsetzung)}}{theorem}
\newtcbtheorem[number within=section]{defi}
    {Definition}{colback=red!7,colframe=red!45!black,fonttitle=\bfseries,title after break={Definition (Fortsetzung)}}{theorem}
\newtcbtheorem[number within=section]{bsp}
    {Beispiel}{colback=white,colframe=blue!70!green,fonttitle=\bfseries, breakable,title after break={Beispiel (Fortsetzung)}}{theorem}
    \newtcbtheorem[number within=section]{b8d}
    {Beachte}{colback=white,colframe=blue!60!green,fonttitle=\bfseries, breakable,title after break={Beachte (Fortsetzung)}}{theorem}
\newtcbtheorem[number within=section]{merke}{Merke} {colback=green!5,colframe=green!35!black,fonttitle=\bfseries, breakable,title after break={Merke (Fortsetzung)}}{theorem}
\newtcbtheorem[number within=section]{bem}
    {Bemerkung}{colback=red!2,colframe=red!35!black,breakable, fonttitle=\bfseries,title after break={Bemerkung (Fortsetzung)}}{theorem}
\usepackage{setspace}
\usepackage{makeidx}
\renewcommand{\indexname}{Sachregister}
\makeindex
\begin{document}

\onehalfspacing
\begin{titlepage}

\begin{center}
%\renewcommand{\baselinestretch}{1}  % Zeilenabstand einfach
\huge \textbf{Gymnasium Gröbenzell}
\vspace{0.2cm}\\
\Large \textbf{ Mathematik Q11}
\vspace{1cm}\\
\LARGE \textbf{Abiturvorbereitung Mathematik}
\vspace{1cm}\\
\end{center}
\end{titlepage}
\tableofcontents
\newpage
\chapter{Analysis}
\begin{section}{Lokales Differenzieren}\index{Differenzieren!lokal}
\begin{subsection}{Die mittlere Änderungsrate}
\begin{defi}{Die mittlere Änderungsrate}{maer}\index{Änderungsrate!mittlere}
Der Quotient $$\dfrac{\Delta y}{\Delta x} = \dfrac{f(b) - f(a)}{b - a}$$ heißt mittler Änderungsrate oder Differenzenquotient der Funktion f im Intervall $\left[a;b\right]$.\\
Geometrisch gedeutet ist der Differenzenquotient \index{Änderungsrate!Differenzenquotient} die Steigung der Sekante durch die Punkte $P(a|f(a))$ und $Q(b|f(b))$.
\end{defi}
\begin{bsp}{Ein Beispiel für die mittlere Änderungsrate}{bspmaer}
Gegeben sind die Punkte $P(3|4)$ und $Q(6|1)$ und damit die mittlere Änderungsrate: $$ \dfrac{\Delta y}{\Delta x} = \dfrac{4-1}{3-6} = \dfrac{3}{-3} = -1$$
\end{bsp}
\begin{bsp}{Graphische Darstellung der Sekantensteigung}{}
Gegeben ist die Funktion $f(x) =\dfrac{1}{2}x^3-x^2-2*x^1 +1$ mit $\mathds{D}_f =\mathds{R}$ und die Punkte $P(1|-1,51)$ und $Q(3,3|1,48)$. Um die mittlere Änderungsrate zu bestimmen muss jetzt die Steigung der Sekante berechnet werden.
$$m=\dfrac{\Delta y}{\Delta x} = \dfrac{y_2 -y_1}{x_2 -x_1} = \dfrac{1,48 -(-1,51)}{3,3 -1} = \dfrac{2,99}{2,3} = \dfrac{13}{10} = 1,3 $$
\begin{center}
\begin{tikzpicture}
\begin{axis}[xmin= -3.5, xmax = 4.5, ymin= -4, ymax= 4,
        axis lines = middle, 
        ymajorgrids=true,
        xmajorgrids=true,
        xtick={-3, -2, -1, 0, ..., 4},
        ytick={-4, -3, -2, ..., 3},
        xlabel = $x$,
        ylabel=$y$]
            \addplot[color= black, samples = 300, domain= -2:3.5]{0.5*x^3-x^2-2*x^1 +1};
           \draw[color = red] (1,-1.51)--(3.3,1.48);
           \draw[color = red] (3.3,-1.51)--(3.3,1.48);
           \draw[color = red] (1,-1.51)--(3.3,-1.51);
           \draw (1,-1.51)-- ++(-2.5pt,-2.5pt) -- ++(5pt,5pt) ++(-5pt,0pt) -- ++(5pt,-5pt) node[left] {P};
           \draw (3.3,1.48)-- ++(-2.5pt,-2.5pt) -- ++(5pt,5pt) ++(-5pt,0pt) -- ++(5pt,-5pt) node[right] {Q};
           \draw (3.3,-1.51)-- ++(-2.5pt,-2.5pt) -- ++(5pt,5pt) ++(-5pt,0pt) -- ++(5pt,-5pt);
            \draw[color = red] (2.2,0.5) node[left] {$\dfrac{\Delta y}{\Delta x} = 1,3$};
\end{axis}
\end{tikzpicture}
\end{center}
\end{bsp}
\end{subsection}
\subsection{Definition der Ableitung}
Die mittlere Änderungsrate beschreibt die durchschnittliche Steigung des Graphen einer Funktion zwischen zwei bestimmten Punkten. Die Ableitung realisiert den Übergang von der Sekante zur Tangente.
\begin{merke*}{Wichtige Informationen zur Definition  der Ableitung}
\index{Ableitung!Definition}
\begin{itemize}
 \item Nähern sich die Funktionswerte $f(x)$ einer Funktion $f$ einer Zahl $a$ beliebig genau an, das ist der Fall wenn sich die Werte für $x$ von links und von rechts an $x_0$ beliebig genau annnähern, so heißt $a$ Grenzwert der Funktion $f(x)$. Kurzschreibweise: $\lim_{x \rightarrow x_0} f(x)= a$ 
 \item Der Punkt Q laufe auf einer Kurve von links und rechts auf den Punkt P zu. Ergibt sich dabei ein gemeinsamer Grenzwert der Sekante, heißt die Grenzgerade Tangente an die Kurve im Punkt P.
 \item Die Steigung eines Graphen im Punkt P ist gleich der Steigung der Tangente in diesem Punkt.
\end{itemize}
\end{merke*}
Um die Ableitung zu definieren, betrachten wir die Normalparabel mit der Gleichung $$ y=x^2 $$ und darauf einen Punkt $P(1,5|2,25)$. Zur Bestimmung der Sekantensteigung bildet man den Differenzenquotienten zwischen dem Punkt $P$ und einem weiteren beliebigen Punkt $Q(x|x^2)$. 
\begin{bem*}{Differenzenquotient für $f(x) =x^2 $}\index{Änderungsrate!Differenzenquotient}
  
  $$ m_s(x)= \dfrac{\Delta y}{\Delta x} = \dfrac{f(x) - 2,25}{x - 1,5} = \dfrac{x^2 - 2,25}{x- 1,5} = \dfrac{(x-1,5)(x+1,5)}{x-1,5} =x+1,5 $$
Nähert sich der Punkt $Q$ beliebig nahe an den Punkt $P$ an, so unterscheidet sich die Steigung der Sekante immer weniger von 3. Man spricht in diesem Fall auch von einem Grenzwert.
$$\lim_{x\rightarrow 1,5} m_s(x) = 3$$
\end{bem*}

\label{hmethode}
\begin{merke*}{Die h-Methode zur Berechnung der Steigung }{}\index{Ableitung!h-Methode}
Wir führen eine Hilfsvariable h ein, diese gibt die Abweichung der Koordinate $x$ von der Koordinate $x_0$ angibt. Damit ist $x=x_0 + h$. Mit dieser Überlegung ist es jetzt möglich den Grenzwert zu bestimmen: $$m_t =\lim_{h\rightarrow 0} m_t(x_0 +h) =\lim_{h\rightarrow 0} \dfrac{f(x_0 + h) - f(x_0)}{h}.$$
\end{merke*}
\begin{defi}{}{} \index{Ableitung!Definition}
    Der Grenzwert des Differenzenquotienten $\dfrac{\Delta y}{\Delta x}$ für $x\rightarrow x_0$ heißt Differentialquotient oder Ableitung der Funktion $f$ an der Stelle $x_0$ und wird mit $f'(x_0)$ bezeichnet. Es gilt der Zusammenhang:$$f'(x_0) = \lim_{x\rightarrow x_0} \dfrac{f(x) - f(x_0)}{x - x_0} = \lim_{h\rightarrow 0} \dfrac{f(x_0 +h) - f(x_0)}{h}$$
\end{defi}
\begin{bem*}{}{}
Mit Hilfe der h-Methode ist es damit möglich jede beliebige Funktion abzuleiten. Man spricht in dem Fall von der Ableitung an einer bestimmten Stelle $x_0$. 
\end{bem*}
\begin{bsp}{Graphische Interpretation der h-Methode}{}
  \begin{center}
\begin{tikzpicture}
\begin{axis}[xmin= -0.2, xmax = 3.5, ymin= -1.5, ymax= 5.5,
        axis lines = middle, 
        ymajorgrids=true,
        xmajorgrids=true,
        xticklabels=\empty,
        yticklabels=\empty,
        xlabel = $x$,
        ylabel=$y$]  
        \addplot[color= black, samples = 300, domain= -2:2.5]{x^2};
        \draw[color= red] (1,-0.2) node[right] {$x_0$};
        \draw (1,0)-- ++(-2.5pt,-2.5pt) -- ++(5pt,5pt) ++(-5pt,0pt) -- ++(5pt,-5pt);
        \draw[color= red] (2,-0.2) node[right] {$x_0 +h$};
        \draw (2,0)-- ++(-2.5pt,-2.5pt) -- ++(5pt,5pt) ++(-5pt,0pt) -- ++(5pt,-5pt);
        \draw[color= red] (0,1.3) node[right] {$f(x_0)$};
        \draw (0,1)-- ++(-2.5pt,-2.5pt) -- ++(5pt,5pt) ++(-5pt,0pt) -- ++(5pt,-5pt);
        \draw[color= red] (0,4.3) node[right] {$f(x_0 + h)$};
        \draw (0,4)-- ++(-2.5pt,-2.5pt) -- ++(5pt,5pt) ++(-5pt,0pt) -- ++(5pt,-5pt);
         \draw[color= red] (2,4) node[right] {$Q$};
        \draw (2,4)-- ++(-2.5pt,-2.5pt) -- ++(5pt,5pt) ++(-5pt,0pt) -- ++(5pt,-5pt);
         \draw[color= red] (1,1) node[right] {$P$};
        \draw (1,1)-- ++(-2.5pt,-2.5pt) -- ++(5pt,5pt) ++(-5pt,0pt) -- ++(5pt,-5pt);
            \draw[red, dashed] (0,1 )-- (2,1);
            \draw[red, dashed] (2,0) -- (2,4);
            \draw[red, dashed] (0,4) -- (2,4);
            \draw[red, dashed] (1,0) -- (1,1);
         \draw[color= red] (1.5,0.7) node[right] {$h$};
         \addplot[color= red,dashed, samples = 300, domain= 0.5:2.5]{3*x - 2};
         \draw (1.4,-0.5) node[right] {$\Longleftarrow$};
         \draw[red] (1.3,-0.8) node[right] {$h\longrightarrow 0$};
\end{axis}
\end{tikzpicture}
\end{center}
 Das bedeutet, dass sich die beiden Punkten der Sekante beliebig genau annähern, ihr Abstand geht damit gegen Null ($h\rightarrow 0$). Somit wird aus der Sekannte eine Tangente.
\end{bsp}
\index{Änderungsrate!lokale}
\begin{bem*}{Tangentensteigung}{}
   Die Ableitung $f'(x_0)$ ist die Steigung der Tangente an den Graphen von $f$ an der Stelle $x_0$. Die Ableitung ist hierbei ein Maß dafür, wie stark sich die Funktionswerte in einer Umgebung um die Stelle $x_0$ ändern. Bei zeitlichen  Vorgängen spricht man von einer \textcolor{red}{momentanen Änderungsrate}, bei anderen Vorgängen von einer \textcolor{red}{lokalen Änderungsrate}.
\end{bem*}
\index{Änderungsrate!Steigungswinkel}
\index{Steigungswinkel}
\begin{bem*}{Der Steigungswinkel}{}
Der \textcolor{red}{Steigungswinkel} $\alpha$ einer Tangente ist der \textcolor{red}{spitze} Winkel zur x-Achse. Der Tangens des Steigungswinkels $\alpha$ ist gleich der Steigung $m$:$$\tan{(\alpha)} = m$$ \begin{center}
    bzw.
\end{center} $$\alpha = \arctan{(m)} = \tan^{-1}{(m)}$$
ist $m$ positiv, ist $\alpha$ positiv; ist $m$ negativ, ist $\alpha$ negativ. 
\begin{center}
\begin{tikzpicture}
\begin{axis}[xmin= -2, xmax = 3.5, ymin= -1.5, ymax= 3.5,
        axis lines = middle, 
        xticklabels=\empty,
        yticklabels=\empty,
        xlabel = $x$,
        ylabel=$y$]
        \addplot[red, samples = 100]{x^3+1};
        \addplot[blue, samples = 100]{3*x-1};
        \draw[color= red] (0.8,2.5) node[right] {$P$};
          \draw (1,2)-- ++(-2.5pt,-2.5pt) -- ++(5pt,5pt) ++(-5pt,0pt) -- ++(5pt,-5pt);

          \draw (1,0) coordinate (a) -- (0.33,0) coordinate (b) -- (1,2) coordinate (c)  pic[ draw = blue, fill = red!25!white, ->, angle eccentricity =1.2, angle radius = 1cm] {angle = a--b--c};
        \draw[color= black] (1,0.6) node[right] {$\alpha$};
\end{axis}
\end{tikzpicture}
\end{center}
\end{bem*}
Mit Hilfe dieser Definitionen und Überlegungen ist es nun möglich die Steigung einer Funktion an einer bestimmten Stelle zu berechnen und den dazugehörigen Steigungswinkel der Tangente zu bestimmen. Anhand eines Beispiels wird dieses jetzt exempkarisch durchgeführt.
\begin{bsp*}{Berechnung der Steigung und Steigungswinkel}{}
Gegeben ist die Funktion $f(x)= 2x^2 - 3x +2 $ und dem Definitionsbereich $\mathds{D}_f = \mathds{R}$. Wir betrachten die Stelle $x_0 = \dfrac{3}{2}$.
\begin{enumerate}
    \item Bestimmung der ersten Ableitung mit Hilfe der h-Methode
    \item Berechnung der Steigung an einer bestimmten Stell $x_0$
    \item Berechnung des Steigungswinkels der Tangente an dieser Stelle am Graphen
\end{enumerate}
\begin{equation*}
\begin{split}
f'(x_0) &= \lim_{h\rightarrow 0} \dfrac{f(x_0+h)-f(x_0)}{h}\\ f'(1,5)&=\lim_{h\rightarrow 0} \dfrac{f(1,5 +h) - f(1,5)}{h}\\ &= \lim_{h\rightarrow 0} \dfrac{2\cdot (1,5 +h)^2 -3\cdot(1,5+h) + 2-(2\cdot (1,5)^2 - 3\cdot (1,5) + 2)}{h}\\
&= \lim_{h\rightarrow 0} \dfrac{2\cdot (2,25 +3h + h^2) -3\cdot(1,5+h) + 2 - (4,5 -2,25 +2)}{h}\\
&= \lim_{h\rightarrow 0} \dfrac{4,5+6h+2h^2-2,25-3h+2-(4,5 -2,25 +2)}{h}\\
&= \lim_{h\rightarrow 0} \dfrac{2h^2+3h}{h}= \lim_{h\rightarrow 0} \dfrac{h\cdot( 2h+3)}{h} = \lim_{h\rightarrow 0} 2h+3\\
&= \textcolor{red}{3 = m_T} \hspace{0.3cm} \longrightarrow \hspace{0.3cm} \textbf{Steigung der Tangente}
\end{split}
\end{equation*}
Berechnung des Steigungswinkels der Tangente an der Stelle $x=1,5 $: 
\begin{equation*}
\begin{split}
\tan{(\alpha)} &= m\\
\tan{(\alpha)} &= 3\\
\alpha &= \arctan{(3)} \approx 71,57^{\circ}  
\end{split}
\end{equation*}
\end{bsp*}
\subsection{Differenzierbarkeit und Lineare Transformation}\index{Differenzieren!Differenzierbarkeit} \index{Ableitung!Differenzierbarkeit}
\begin{defi}{Die Differenzierbarkeit einer Funktion}{}
 Eine Funktion f heißt an der Stelle $x_0$ im Inneren des Definitionsbereichs $D_f$ differenzierbar, wenn der Grenzwert \begin{equation*}
     f'(x_0) =\lim_{h\rightarrow 0}\dfrac{f(x_0 +h) -f(x_0)}{h}
 \end{equation*} existiert, sich also bei links- und rechtsseitiger Annäherung der gleiche Wert ergibt. Der Grenzwert des Differenzenquotient heißt Differentialquotient.
\end{defi}
\begin{bsp*}{Die Überprüfung der Differenzierbarkeit mit der h-Methode}\index{Differenzieren!Differenzierbarkeit!Beispiel}
    Gegeben ist die Funktion $f(x)= x^2 + |x|$, wir überprüfen die Ableitung an der Stelle $P(0|f(0))$.

$$f(x)= \begin{cases}
              x^2 + x & \text{für}\hspace{0.5cm} x\geq 0\\
              x^2 - x & \text{für}\hspace{0.5cm}  x< 0 
         \end{cases}$$
Für die Untersuchung der Differenzierbarkeit sind folgende Grenzwerte zu überprüfen::
\begin{equation*}
\begin{split}
f'(0) &= \lim_{h\rightarrow0^-} \dfrac{f(0+h)-f(0)}{h}\\ &=\lim_{h\rightarrow 0^-} \dfrac{(0+h)^2 - (0+h) -0}{h}\\ &= \lim_{h\rightarrow 0^-} \dfrac{h^2-h}{h} = \lim_{h\rightarrow 0^-} h-1=-1 
\end{split}
\end{equation*}
\begin{equation*}
\begin{split}
f'(0)&=\lim_{h\rightarrow 0^+}\dfrac{f(0+h)-f(0)}{h}\\
&=\lim_{h\rightarrow 0^+} \dfrac{(0+h)^2 + (0+h) -0}{h}\\ 
&=\lim_{h\rightarrow 0^+} \dfrac{h^2 + h}{h} = \lim_{h\rightarrow 0^+} h+1 =1
\end{split}
\end{equation*}
Der Differenzenquotient hat daher für $x\rightarrow 0$ keinen Grenzwert, die Funktion $f$ ist damit an der Stelle $x_0 = 0$ nicht differenzierbar. Der Graph $G_f$ hat an dieser Stelle einen "`Knick"'.
\end{bsp*}

\begin{bem*}{}\index{Differenzieren!Differenzierbarkeit!Lineare Transformationen}\index{Lineare Transformation}
Die Lineare Transformation der Betragsfunktion ändert nichts an der Nicht-Differenzierbarkeit der Funktion.
\end{bem*}
Während der letzten Schuljahre wurden verschiedene Funktionsarten einer linearen Transformation unterzogen. Als Beispiel wird hier auf die Funktionen: 
 \begin{equation*}
    \begin{split}
        f(x) &= x^2\\
        g(x) &= \sin{(x)}\\
        h(x) &= \cos{(x)}
    \end{split}
\end{equation*}
verwiesen.\\ 
In Jahrgangsstufe 9 wurde die Normalparabel auf unterschiedlichste Arten transformiert. In der 10. Jahrgangsstufe wurden Trigonometrischen Funktionen transformiert.
\begin{bsp*}{Transformationen am Bespiel der Normalparabel}{}\index{Lineare Transformation!Beispiel}
Der Graph der Funktion $f(x) = x^2$ läßt sich wie folgt transformieren:
\begin{itemize}
\item $g_1(x) = -x^2$ $\Longrightarrow$ Normalparabel an der $x-$Achse gespiegelt
\item $g_2(x) =   x^2 +a$ $\Longrightarrow$ $G_f$  Richtung der y--Achse um a verschoben
\item $g_3(x) =(x-a)^2$ $\Longrightarrow$ $G_f$ in Richtung der x--Achse um a verschoben
\item $g_4(x) =a\cdot x^2$ und $a>0$, $\Longrightarrow$ $G_f$ in Richtung der y--Achse mit dem Faktor a gestreckt bzw. gestaucht
\end{itemize}
\end{bsp*}
\begin{merke}{Allgemeine Darstellung der Linearen Transformationen} 
Die lineare Transformation lässt sich allgemein wie folgt schreiben:\index{Lineare Transformation!allgemein}\\
Wir erhalten aus dem Graphen $G_f$ der Funktion f den Graphen der Funktion g mit:
\begin{itemize}
\item $g(x) = - f(x)$, indem man $G_f$ an der x--Achse spiegelt
\item $g(x) =f(-x)$, indem man $G_f$ an der y--Achse spiegelt
\item $g(x) = f(x) +a$ indem man $G_f$ in Richtung der y--Achse um a verschiebt
\item $g(x) =f(x-a)$ indem man $G_f$ in Richtung der x--Achse um a verschiebt
\item $g(x) =a\cdot f(x)$ und $a>0$, indem man $G_f$ in Richtung der y--Achse mit dem Faktor a streckt bzw. staucht
\item $g(x) =f(a\cdot x)$ und  $a>0$, indem man $G_f$ in Richtung der x--Achse mit dem Faktor $\dfrac{1}{a}$ staucht bzw. streckt 
\end{itemize}
\end{merke}
\end{section}

\begin{section}{Globales Differenzieren}\index{Differenzieren!global}
\subsection{Die Ableitungsfunktion}
\begin{defi}{Die Ableitungsfunktion}{ablf}
    \index{Ableitung!Ableitungsfunktion}
    Zu einer Funktion $f$ heißt die Funktion $f': x \mapsto f'(x)$ die Ableitungsfunktion oder kurz die Ableitung von $f$. Die Ableitung einer Funktion $f$ ist diejenige Funktion, die jeder Stelle $x$, an der $f$ differenzierbar ist, die Ableitung $f'(x)$ zuordnet:
$$f'(x)=\lim_{h\rightarrow 0} \dfrac{f(x + h) - f(x)}{h}.$$ 
\end{defi}
\begin{bem}{Bedeutung der Ableitungsfunktion}{bedeutungAblf}
   Die Ableitungsfunktion $f'$ beschreibt die Steigung der Funktion $f$ an jeder Stelle $x\in \mathds{R}$. \\
   Bei der Ableitung von Polynomen verringert sich der Grad der Funktion bei jeder Ableitung um 1.\\
   
\end{bem}
\begin{bsp}{Beispielfunktion}{}\index{Ableitung!Ableitungsfunktion!Beispiel}
Die Funktion $f(x) = x^3 + x^2 -2x+1$ mit $x\in\mathds{R}$ ist ein Polynom vom Grad 3, die Funktion $f'(x) = 3x^2+2x-2$ ist die erste Ableitung der Funktion $f(x)$. Die Graphen der beiden Funktionen ist unten gegeben. 
\definecolor{ccqqqq}{rgb}{0.8,0,0}
\begin{center}
\begin{tikzpicture}[line cap=round,line join=round,>=triangle 45,x=1cm,y=1cm]
\begin{axis}[
x=1cm,y=1cm,
axis lines=middle,
ymajorgrids=true,
xmajorgrids=true,
xmin=-4.6725325047810795,
xmax=3.4865844816315783,
ymin=-3.970650919871985,
ymax=4.84273488148887,
xtick={-4,-3,...,3},
ytick={-3,-2,...,4},]
\clip(-4.6725325047810795,-3.970650919871985) rectangle (3.4865844816315783,4.84273488148887);
\draw[line width=1.2pt,smooth,samples=100,domain=-4.6725325047810795:3.4865844816315783] plot(\x,{(\x)^(3)+(\x)^(2)-2*(\x)+1});
\draw[line width=1.2pt,color=ccqqqq,smooth,samples=100,domain=-4.6725325047810795:3.4865844816315783] plot(\x,{3*(\x)^(2)+2*(\x)-2});
\begin{scriptsize}
\draw[color=black] (-2,-2.7) node {$f(x)$};
\draw[color=ccqqqq] (-2.2,3.5) node {$f'(x)$};
\end{scriptsize}
\end{axis}
\end{tikzpicture}
\end{center}
\end{bsp}
\subsection{Ableitung ganzrationaler Funktionen}
Die Ableitung von Funktionen mit der h-Methode ist für jede Funktion möglich. Allerdings wird das Ableiten damit schnell mühsam. Durch die Anwendung von Regeln zur Ableitung ganzrationaler Funktionen wird das Bilden der ersten Ableitung um einiges einfacher.\\
\begin{merke*}{Ableitungsregel Potenzfunktionen}{}\index{Ableitung!Regel!Potenzfunktion}
Die Potenzfunktion $f$ mit $f(x) = x^n$ und natürlichen Exponenten $n$ ist differenzierbar und es gilt: $$f'(x) = n\cdot x^{n-1}$$
\underline{Merkregel:} Der Term $x^n$ wird differenziert, indem man den Exponenten $n$ als Faktor voranstellt und den Exponenten um 1 verringert.
\end{merke*}
\begin{bsp}{Beispiele für das Ableiten von Potenzfunktionen}{}
\begin{itemize}
    \item $f(x)= x^3\ \longrightarrow f'(x) = 3x^2$
    \item $g(x) = \dfrac{3}{4} x^5 \ \longrightarrow g'(x)=  5\cdot \dfrac{3}{4} x^4 = \dfrac{15}{4}x^4$ 
    \item $h(x) = x^{\pi} \ \longrightarrow h'(x) = \pi \cdot x^{\pi -1}$
\end{itemize}
\end{bsp} 
\begin{defi}{Die Normale}{}\index{Normale}
    Die Gerade, die im Berührpunkt auf der Tangente senkrecht steht, heißt Normale. Hierbei hängen die Steigung der Normalen und die Steigung der Tangente wie folgt zusammen $$m_{Normale} = -\dfrac{1}{m_{Tangente}}$$
\end{defi}

\subsection{Anwendung der Ableitungsfunktion-Newton-Verfahren}\index{Newton-Verfahren}

Innerhalb der Analysis ist die Bestimmung der Nullstellen eine der zentralen Aufgaben. Diese Berechnung erfolgt allerdings nur bei einer begrenzten Anzahl von Funktionsklassen problemlos. So gibt es bei Polynomen vom Grad $n\geq 4$ keine Lösungsformel\footnote{Für den Grad $n=2$ ist die Lösungsformel die bereits bekannte Mitternachtsformel: $x_{1,2} = \dfrac{-b\pm \sqrt{b^2 -4ac}}{2\cdot a}$} mehr. Für den Grad $n\geq 4$ gibt es bei Polynomen dann verschiedene Wege die Nullstellen zu bestimmen. Eine dieser Möglichkeiten ist die Polynomdivision\footnote{Die Polynomdivision ist nicht Teil des Lehrplans der Klasse 11, allerdings aus der 10. Klasse bekannt.}, siehe hierzu auch \ref{polynomdivision}. Eine weitere Möglichkeit die Nullstellen zu bestimmen liefert das Newton-Verfahren. Hierbei ist es möglich, die Nullstellen beliebig genau näherungsweise zu bestimmen. \\
Funktionsweise des Newton-Verfahren:
\begin{merke*}{Iterationsverfahren}\index{Iterationsverfahren}
  Für die Durchführung des Newton-Verfahrens werden bestimmte Rechenschritte beliebig oft wiederholt. Da sich dabei die Rechenschritte nur mit sich ändernden Zahlen wiederholen, handelt es sich um ein iteratives Verfahren.
\end{merke*}
Um das Newton-Verfahren zu verstehen, betrachtet man die allgemeine Herleitung. Ziel des Verfahren ist es, eine Nullstelle einer beliebigen Funktion $f$ zu finden. Setzt man einen Anfangswert $x_0$ in die Funktionsgleichung ein, erhält man im Allgemeinen einen von Null verschiedenen Funktionswert $f(x_0)$. Bildet man jetzt die Tangente durch den Punkt $A(x_0|f(x_0))$ am Graphen $G_f$ der Funktion $f$ und bestimmt davon die Nullstelle erhält man im Allgemeinen eine bessere Näherung für die Nullstelle des Graphen $G_f$\footnote{Ein Besipiel für das Bestimmen einer Tangentengleichung findet man im Beispiel \ref{Tangentengleichung}}. 
\begin{bsp*}{Herleitung des Newton-Verfahren}\index{Newton-Verfahren!Herleitung}
    Gegeben ist eine beliebige Funktion $f$ und ein Startwert $x_0 \in \mathds{D}_f$ und $A(x_0|f(x_0))$ mit $f'(x_0)\neq 0$. Für die Tangente durch $x_0$ an $G_f$ gilt allgemein folgende Geradengleichung $y_T= m\cdot x + t$. Am Punkt $A(x_0|f(x_0))$ ergibt sich damit folgende Gleichung:$$y_T = f'(x_0) \cdot x + t.$$ Um die Nullstelle dieser bestimmten Tangente bestimmen zu können, ist es notwendig, dass man den y-Achsenabschnitt bestimmt. Hierfür werden die Werte $x_0$ und $f(x_0)$ in die Geradengleichung eingesetzt und nach $t$ aufgelöst. 
    \begin{equation*}
	\left.\begin{aligned}
	f(x_0) &= f'(x_0) \cdot x_0 + t \\
	t &= f(x_0) - f'(x_0) \cdot x_0 
	\end{aligned}
	\right\}
\quad \text{Tangentengleichung:}\hspace{0.1cm } y_T = f'(x_0) \cdot x + (f(x_0) - f'(x_0) \cdot x_0 )
\end{equation*}
Um die erste Näherung zu bekommen, wird die Nullstelle der Tangentengleichung berechnet.
    \begin{equation*}
	\left.\begin{aligned}
	0 &=  f'(x_0) \cdot x + (f(x_0) - f'(x_0) \cdot x_0 ) \\
	f'(x_0) \cdot x &= f'(x_0)\cdot x_0  - f(x_0)\\
    x &=\dfrac{f'(x_0)\cdot x_0  - f(x_0)}{f'(x_0)}
	\end{aligned}
	\right\}
\quad \text{Näherungswert:}\hspace{0.1cm } x = x_0 -\dfrac{f(x_0)}{f'(x_0)}
\end{equation*}
\end{bsp*}
\begin{defi}{Das Newton-Verfahren}{}\index{Newton-Verfahren!Definition}
    Ist $x_n$ ein Näherungswert einer Nullstelle der Funktion $f$ und $f'(x_n)\neq 0$, dann ist $$x_{n+1} = x_n -\dfrac{f(x_n)}{f'(x_n)}$$ im Allgemeinen ein besserer Näherungswert.
\end{defi}
\begin{bem}{Konvergenz des Newton-Verfahren}{}\index{Newton-Verfahren!Konvergenz}
    Das Newton-Verfahren konvergiert nicht immer. Die Konvergenz der Iteration hängt vom Startwert $x_0$ ab. Hierbei bedeutet der Begriff Konvergenz, dass sich die Näherungswerte der eigentlichen Nullstelle immer genauer annähern.
\end{bem}
\begin{bsp}{Die Funktion $f(x)= x^3 -2x^2-x+1$}{}
Am Beispiel der Funktion $f(x)=x^3-2x^2-x+1$ soll das Newton-Verfahren exemplarisch durchgeführt werden. Als Startwert wird die Stelle $x_0 = 1,24$ gewählt. Um die Iteration durchführen zu können, muss die erste Ableitung $f'$  der Funktion $f$ gebildet werden, diese lautet $f'(x)= 3x^2 -4x -1$. Nun ist es möglich, die einzelnen Iterationschritte durchzuführen.
\begin{center}$\begin{aligned}
	x_n &= x_{n-1}-\dfrac{f(x_{n-1})}{f'(x_{n-1})} \\
	x_0 &= 1,24\\
    x_1 &= x_0 - \dfrac{f(x_0)}{f'(x_0)}\\
    &=1,24 - \dfrac{f(1,24)}{f'(1,24)}\\
    &=1,24 - \dfrac{-1,41}{-1,3472}\\
    &=\dfrac{4093}{21050}\approx 0,19\\
    x_2&= 0,64
	\end{aligned}$\end{center}
\begin{center}
    \begin{tikzpicture}
        \begin{axis}[xmin= -3.5, xmax = 4.5, ymin= -4, ymax= 4,
        axis lines = middle, 
        ymajorgrids=true,
        xmajorgrids=true,
        xtick={-3, -2, -1, 0, ..., 4},
        ytick={-4, -3, -2, ..., 3},
        xlabel = $x$,
        ylabel=$y$]
            \addplot[color= black, samples = 300, domain= -2:3]{x^3-2*x^2-x^1 +1};
            \addplot[color= black, samples = 300, domain= -1:3]{-1.35*x^1 +0.27};
            \draw[dashed] (1.24,-1.41)  -- (1.24,0) node(xline)[above] {$x_0$};
            \draw (1,-1.5)   node[left] {$A$};
            \draw (1.2390888331407657,-1.4073470487766349)-- ++(-2.5pt,-2.5pt) -- ++(5pt,5pt) ++(-5pt,0pt) -- ++(5pt,-5pt);
            \draw[dashed] (0.2,0.2) node(xline)[above] {};
            \draw (0.1,-0.6)   node [above] {$x_1$};
            \draw[dashed] (0.2,0)  -- (0.2,0.73) node(xline)[right] {$f(x_1)$};
            \draw (0.2,0.73)-- ++(-2.5pt,-2.5pt) -- ++(5pt,5pt) ++(-5pt,0pt) -- ++(5pt,-5pt);
        \end{axis}
    \end{tikzpicture}     
\end{center}
\end{bsp}
Die einzelnen Iterationsschritte sind hierbei nicht immer von einzeln durchzuführen. Mit den aktuell erlaubten Taschenrechnern ist es möglich, die sich wiederholenden Schritte immer wieder durchzuführen\footnote{Siehe Anleitung Taschenrechner für die Programmierung.}.
\end{section}
\begin{section}{Kurvendiskussion}
 \subsection{Monotonie und Extrema}   
 \begin{defi}{Monotonie}{}\index{Kurvendiskussion!Monotonie}\index{Monotonie}
 Es sei $f$ eine Funktion auf einem Intervall $I$ definiert. Gilt für alle $x_1, x_2 \in I$ mit $x_1<x_1:$
 \begin{equation*}
	\left.\begin{aligned}
	f(x_1) &<  f(x_2)  \\
	f(x_1) &> f(x_2)
	\end{aligned}
	\right\}
\quad \text{So nennt man f}
	\quad\left\{\begin{aligned}
	\text{monoton steigend}  \\
	\text{monoton fallend}
	\end{aligned}
\right\}
\quad
 \text{im Intervall I} 
\end{equation*} 
 \end{defi}
 \begin{defi}{Extrema}{}
   Es sei $f$ eine auf einem Intervall $I$ definierte Funktion. Dann nennt man $f(x_0)$ \textcolor{red}{lokales Maximum} wenn es eine Umgebung $U$ um $x_0$ gibt, sodass für alle  $x \in U \in \mathds{D_f}$ gilt: \textcolor{red}{$f(x_0)\geq f(x)$}. Der Punkt \textcolor{red}{$P(x_0|f(x_0))$} heißt dann \textcolor{red}{Hochpunkt (HoP)}.\\
   Es sei $f$ eine auf einem Intervall $I$ definierte Funktion. Dann nennt man $f(x_0)$ \textcolor{red}{lokales Minimum} wenn es eine Umgebung $U$ um $x_0$ gibt, sodass für alle  $x \in U \in \mathds{D_f}$ gilt: \textcolor{red}{$f(x_0)\leq f(x)$}. Der Punkt \textcolor{red}{$P(x_0|f(x_0))$} heißt dann \textcolor{red}{Tiefpunkt (TiP)}.
 \end{defi}
 \begin{satz}{Bedingung für die Monotonie}{}\index{Monotonie!Bedingung}
 Ist die Funktion $f$ im Intervall $I$ eine differenzierbare Funktion und 
  \begin{equation*}
	\left.\begin{aligned}
	f'(x) &>0  \\
	f'(x) &<0
	\end{aligned}
	\right\}
\quad \text{für alle $x\in I$, dann ist $f$ in $I$ streng monoton}
	\quad\left\{\begin{aligned}
	\text{wachsend}  \\
	\text{fallend}
	\end{aligned}
\right.
\end{equation*} 
 \end{satz}
 \begin{satz}{Notwendige Bedingung für innere Extrempunkt}{}\index{Monotonie!Notwendige Bedingung}
   Eine Funktion $f$ und ihre Ableitung $f'$ seien auf einem Intervall $I$ definiert. Dann gilt: \textcolor{red}{Hat} $f$ an einer inneren Stelle $x_0\in I$ einen \textcolor{red}{Extremwert}, so ist \textcolor{red}{$f'(x_0)=0$}
 \end{satz}
 \begin{b8d}{Vorzeichenwechsel und Extrema}{}\index{Monotonie!Vorzeichenwechsel}
 Die Art der einzelnen Extrema läßt sich leicht durch die Vorzeichenwechsel (VZW) der Ableitung bestimmen.
 \begin{itemize}
\item Hochpunkt (HoP) $HoP(x_0|f(x_0))$ genau dann, wenn es einen VZW der Ableitung von \textcolor{red}{positiv nach negativ} gibt
\begin{enumerate}
    \item $f'(x_0) = 0$
    \item $x_i<x_0 \Rightarrow f'(x_i)>0$
    \item $x_0 <x_i \Rightarrow f'(x_i)<0$
\end{enumerate}
\item Tiefpunkt (TiP) $TiP(x_0|f(x_0))$ genau dann, wenn es einen VZW der Ableitung von \textcolor{red}{negativ nach positiv} gibt
\begin{enumerate}
    \item $f'(x_0) = 0$
    \item $x_i<x_0 \Rightarrow f'(x_i)<0$
    \item $x_0 <x_i \Rightarrow f'(x_i)>0$
\end{enumerate}
\item Terrassenpunkt (TP) $TP(x_0|f(x_0))$
\begin{enumerate}
    \item $f'(x_0) = 0$
    \item \textcolor{red}{kein VZW}
\end{enumerate}
 \end{itemize}
 \end{b8d}
 \begin{bsp}{Bestimmung der Monotonie und der Extrema I}{}
 Gegeben ist die Funktion $f(x) = 2x^3 +6x^2 -2x + 1$ mit $\mathds{D}_f = \mathds{R}$ und der ersten Ableitung $f'(x) = 6x^2+12x-2$.\\
 Vorgehen:
 \begin{enumerate}
     \item Bestimmung der Nullstelle der ersten Ableitung
     \item Untersuchung der Monotonie mit Hilfe der Monotonietabelle
     \item Entscheidungen zu möglichen Extrema
 \end{enumerate}
 \begin{center}$\begin{aligned}
	f'(x)&= 6x^2+12x-2\\
    0 &= 6x^2+12x-2\\
    x_{1,2} &= \dfrac{-12 \pm \sqrt{144 +48} }{12} = \dfrac{-12\pm 8\sqrt{3}}{12}\\
    &= \dfrac{-3\pm 2\sqrt{3}}{3}\\
    x_1&= \dfrac{-3+ 2\sqrt{3}}{3}\\
    x_2&= \dfrac{-3 - 2\sqrt{3}}{3}
	\end{aligned}$\end{center}
 Mit diesen beiden Nullstellen ist es möglich eine faktorisierte Form des Funktionsterms aufzuschreiben. $$f'(x)= 6x^2+12x-2 = 6\cdot \left(x-x_1 \right)\cdot \left(x-x_2 \right)$$
Monotonietabelle:\\
\begin{center}\begin{tabular}{||c|c|c|c|c|c||}
    \hline
    $x$& $ -\infty <x<x_2 $ & $ x_2 \approx -2.15$ &$ x_2<x<x_1 $ & $x_1 \approx 0,15 $& $ x_1<x<\infty $\\
    \hline \hline
    $6\cdot (x-x_2)$ & - &  0 & - &  & +  \\
    \hline
    $(x -x_1)$ & - & & + &0 & + \\
    \hline
    $f'(x)$ & + & 0 & - & 0 & +\\ 
    \hline
    $G_f$ & smw & HoP & smf & TiP & smw\\
    \hline
\end{tabular}
\end{center}
\begin{center}
    \begin{tikzpicture}
        \begin{axis}[xmin= -4.1, xmax = 2.5, ymin= -15.1, ymax= 20.5,
        axis lines = middle, 
        ymajorgrids=true,
        xmajorgrids=true,
        xtick={-4, -3.5, -3,  ..., 2},
        ytick={-15, -10, -5, ..., 20},
        xlabel = $x$,
        ylabel=$y$]
            \addplot[color= red, samples = 300, domain= -3.7:1.7]{2*x^3+ 6*x^2 -2*x+1};
         \draw (1.5,15)   node [right] {\textcolor{red}{$f(x)$}};  
            \addplot[color= black, samples = 300, domain= -3.7:1.7]{6*x^2+ 12*x -2};
            \draw (-3.5,12)   node [right] {$f'(x)$}; 
            \draw (-2.1,0)   node [above] {$x_2$};
            \draw (0.15,0.5)   node [left] {$x_1$};
             \draw (-2.15,0)-- ++(-2.5pt,-2.5pt) -- ++(5pt,5pt) ++(-5pt,0pt) -- ++(5pt,-5pt);
             \draw (-2.1,14)   node [above] {HoP};
              \draw (-2.15,13.15)-- ++(-2.5pt,-2.5pt) -- ++(5pt,5pt) ++(-5pt,0pt) -- ++(5pt,-5pt);
               \draw (0.15,0.84)-- ++(-2.5pt,-2.5pt) -- ++(5pt,5pt) ++(-5pt,0pt) -- ++(5pt,-5pt);
               \draw (0.15,4.5)   node [above] {TiP};
              \draw (0.15,0)-- ++(-2.5pt,-2.5pt) -- ++(5pt,5pt) ++(-5pt,0pt) -- ++(5pt,-5pt);
        \end{axis}
    \end{tikzpicture}  
\end{center}
\end{bsp}
\begin{bsp}{Bestimmung der Monotonie und der Extrema II}{}
Betrachtet wird die Funktion $f(x)= x^3-6x^2 +9x-2$ mit $\mathds{D}_f =\mathds{R}$ und der ersten Ableitung $f'(x)= 3x^2 -12x +9$
\begin{center}$\begin{aligned}
	f'(x)&= 3x^2 -12x +9\\
    0 &= 3x^2 -12x +9\\
    x_{1,2} &= \dfrac{12 \pm \sqrt{144 -8} }{6} = \dfrac{12\pm \sqrt{36}}{6} = \dfrac{12 \pm 6}{6} = 2\pm 1\\
    x_1&= 3\\
    x_2&= 1
	\end{aligned}$\end{center}
 Mit diesen beiden Nullstellen ist es möglich eine faktorisierte Form des Funktionsterms aufzuschreiben. $$f'(x)= 3x^2-12x+9 = 3\cdot \left(x-3 \right)\cdot \left(x-1 \right)$$

Monotonietabelle:\\
\begin{center}\begin{tabular}{||c|c|c|c|c|c||}
    \hline
    $x$& $ -\infty <x<1 $ & $ x_2 =1$ &$ 1<x<3 $ & $x_1 = 3 $& $ 3<x<\infty $\\
    \hline \hline
    $3\cdot (x-1)$ & - &  0 & - &  & +  \\
    \hline
    $(x -3)$ & - & & + & 0 & + \\
    \hline
    $f'(x)$ & + & 0 & - & 0 & +\\ 
    \hline
    $G_f$ & smw & $HoP(1|2)$ & smf & $TiP(3|-2) $& smw\\
    \hline
\end{tabular}
\end{center}
\begin{center}
    \begin{tikzpicture}
        \begin{axis}[xmin= -1.1, xmax = 5.5, ymin= -4.1, ymax=3.5,
        axis lines = middle, 
        ymajorgrids=true,
        xmajorgrids=true,
        xtick={-4, -3, -2,  ..., 5},
        ytick={-4, -3, -2, ..., 3},
        xlabel = $x$,
        ylabel=$y$]
            \addplot[color= red, samples = 300, domain= -0.2:4.2]{x^3-6*x^2+9*x-2};
     \addplot[color= black, samples = 300, domain= -0.2:4.2]{3*x^2-12*x+9};
            \draw (4.2,2)   node [right] {$f(x)$}; 
            \draw (1,3)   node [right] {$f'(x)$}; 
            \draw (1.2,0)   node [above] {$x_2$};
            \draw (2.8,0)   node [above] {$x_1$};
             \draw (1,0)-- ++(-2.5pt,-2.5pt) -- ++(5pt,5pt) ++(-5pt,0pt) -- ++(5pt,-5pt);
              \draw (3,0)-- ++(-2.5pt,-2.5pt) -- ++(5pt,5pt) ++(-5pt,0pt) -- ++(5pt,-5pt);
               \draw (1,2.2)   node [above] {HoP};
                \draw (1,2)-- ++(-2.5pt,-2.5pt) -- ++(5pt,5pt) ++(-5pt,0pt) -- ++(5pt,-5pt);
                 \draw (3,-2)-- ++(-2.5pt,-2.5pt) -- ++(5pt,5pt) ++(-5pt,0pt) -- ++(5pt,-5pt);
               \draw (3,-2.8)   node [above] {TiP};
        \end{axis}
    \end{tikzpicture}  
\end{center}
\end{bsp}
\begin{bem}{Tangentengleichungen}{}
Eine klassische Aufgabe innerhalb der Kurvendiskussion bildet die Bestimmung einer Tangente an einen beliebigen Punkt des Graphen. Hierfür ist es notwendig folgende Informationen zu bestimmen:
\begin{enumerate}
    \item Bildung der ersten Ableitung
    \item Berechnung der Koordinaten des Punktes auf dem Graphen
    \item Bestimmung der Steigung des Graphen in diesem Punkt
    \item Aufstellen der Geradengleichung der Tangente
\end{enumerate}
\end{bem}
\label{Tangentengleichung}
\begin{bsp}{Bestimmung der Tangentengleichung}{}\index{Kurvendiskussion!Tangentengleichung}
  Betrachtet wird die Funktion $f(x)= x^3 -6x^2 +9x -2$ mit $\mathds{D}_f = \mathds{R}$ und der Punkt $P(2|0)$.
 \begin{center}$\begin{aligned}
	f'(x)&= 3x^2 -12x +9\\
    f'(2) &= 3\cdot 2^2 -12 \cdot 2 +9 = -3
	\end{aligned}$\end{center} 
 Aufstellen der Tangentengleichung aus den berechneten Werten $P(2|0)$ und $m= f'(2) = -3$: 

\begin{equation*}
	\left.\begin{aligned}
	0 &=  f'(2) \cdot 2 + t \\
	0 &= -3 \cdot 2 +t\\
    t &=6
	\end{aligned}
	\right\}
\quad \text{damit folgt für die Tangentengleichung:}\hspace{0.1cm } y_T = -3\cdot x +6 
\end{equation*}
\end{bsp}
\subsection{Die Extremstellen der lokalen Änderungsrate - Wendestellen} \index{Krümmungsverhalten} \index{Kurvendiskussion!Krümmungsverhalten}
\begin{defi}{Der Wendepunkt}{}\index{Krümmungsverhalten!Wendepunkt}\index{Kurvendiskussion!Wendepunkt}
Der Punkt, an dem sich die Krümmung des Graphen der Funktion $f$ ändert, heißt Wendepunkt. Am Wendepunkt des Graphen liegt ein Extremwert der lokalen Änderungsrate vor. Ein Terrassenpunkt ist ein Wendepunkt mit einer waagerechten Tangente.
\end{defi}

\begin{satz}{Zusammenhang der lokalen Änderungsrate und der Krümmung}{krümmung}
  Wenn die lokale Änderungsrate $f'$ einer differenzierbaren Funktion $f$ im Intervall $I$ \textcolor{red}{streng monoton zunimmt}, dann ist der Graph $G_f$ \textcolor{red}{linksgekrümmt}.  \\
  Wenn die lokale Änderungsrate $f'$ einer differenzierbaren Funktion $f$ im Intervall $I$ \textcolor{red}{streng monoton abnimmt}, dann ist der Graph $G_f$ \textcolor{red}{rechtsgekrümmt}. 
\end{satz}
\begin{satz}{Kriterium der Krümmung}{}
Ist die Funktion $f$ im Intervall $I$ zweimal stetig differenzierbar und ist für alle $a\in I$ der Funktionswert $f''(x)$ \textcolor{red}{positiv}, dann ist der Graph der Funktion $f$ \textcolor{red}{linksgekrümmt}.\\
Ist die Funktion $f$ im Intervall $I$ zweimal stetig differenzierbar und ist für alle $a\in I$ der Funktionswert $f''(x)$ \textcolor{red}{negativ}, dann ist der Graph der Funktion $f$ \textcolor{red}{rechtsgekrümmt}.

\index{Krümmungsverhalten!Kriterium}\end{satz}
\begin{bem}{Extremwert der Änderungsrate}{}
Der Punkt, an dem sich die Krümmung des Graphen ändert, heißt \textcolor{red}{Wendepunkt}. Beim \textcolor{red}{Wendepunkt} liegt ein \textcolor{red}{Extremwert} der lokalen Änderungsrate vor. Ein \textcolor{red}{Terrassenpunkt} ist ein \textcolor{red}{Wendepunkt mit einer waagerechten Tangente}.
\end{bem}
Die Krümmung wird, genau wie Monotonieuntersuchung, durch eine Vorzeichentabelle untersucht.  
\begin{bsp}{Untersuchung der Krümmung}{}
Untersucht wird die Krümmung des Graphen der Funktion $f(x) = x^3 -3x^2$ mit $\mathds{D}_f = \mathds{R}$. Bestimmung der ersten und zweiten Ableitung der Funktion f.
\begin{equation*}
    \begin{split}
        f(x) &= x^3 -3x^2\\
        f'(x) &= 3x^2 -6x\\
        f''(x) &= 6x -6 = 6\cdot(x-1)
    \end{split}
\end{equation*}
Berechnung der Nullstelle der zweiten Ableitung:
\begin{equation*}
    \begin{split}
        0 &= 6\cdot(x-1) \hspace{0.2cm} | : 6\\
        0 &= x-1\hspace{0.9cm} | + 1\\
        x &= 1 
    \end{split}
\end{equation*}
\begin{center}\begin{tabular}{||c|c|c|c||}
    \hline
    $x$& $ -\infty <x<1$ & $x = 1$ & $ 1<x<\infty$ \\
    \hline \hline
    $f''(x)=6(x-1)$ & - & 0 & +\\
    \hline
    $G_f$ & rechtsgekrümmt & Wendepunkt $WP(1|-2)$ & linksgekrümmt\\
    \hline
    \hline
\end{tabular}
\end{center}
Berechnung der Koordinaten des Wendepunktes:
\begin{equation*}
    \begin{split}
        f(1) &= (1)^3 -3\cdot (1)^2\\
         &= 1 - 3 \\
         &= -2 
    \end{split}
\end{equation*}
\begin{center}
\begin{tikzpicture}
    \begin{axis}[xmin= -2.1, xmax = 4.1, ymin= -6.1, ymax=6.1,
        axis lines = middle, 
        ymajorgrids=true,
        xmajorgrids=true,
        xtick={-2, -1, -0,  ..., 4},
        ytick={-6, -5, -4, ..., 6},
        xlabel = $x$,
        ylabel=$y$]
        \addplot[color= black, samples = 300, domain= -1.8:3.8]{x^3 -3*x^2};
        \addplot[color= red, samples = 300, domain= -1.8:3.8]{6*x -6};
        \draw (1,-2)-- ++(-2.5pt,-2.5pt) -- ++(5pt,5pt) ++(-5pt,0pt) -- ++(5pt,-5pt);
        \draw (1.2,-2.2)   node [above] {WP};
        \draw[red] (1.5,5)   node [above] {$f''(x)$};
        \draw (3.7,3)   node [above] {$f(x)$};

    \end{axis}
\end{tikzpicture}
\end{center}
\end{bsp}
\subsection{Ausführliche Untersuchung ganzrationaler Funktionen} \index{Kurvendiskussion!Vollständige Untersuchung }
Die ausführliche Kurvendiskussion besteht aus einer Reihe von analytischen Untersuchungen des Terms der Funktion um damit auf das Verhalten des Graphen der Funktion zu schließen. Hierbei ist es allerdings nicht notwendig alle Schritte auswendig zu lernen.
\begin{merke}{Schritte der ausführlichen Kurvendiskussion}{}
\begin{enumerate}
    \item Untersuchungen am ursprünglichen Term
    \begin{itemize}
        \item Bestimmung des maximalen Definitionsbereichs
        \item Untersuchungen der Symmetrie
        \item Verhalten an den Rändern des Definitionsbereichs 
        \item Gemeinsame Punkte mit den Koordinatenachsen
    \end{itemize}
    \item Untersuchungen des Terms der ersten Ableitung
        \begin{itemize}
            \item Monotonie
            \item Extrempunkte

        \end{itemize}
    \item Untersuchungen des Terms der zweiten Ableitung
        \begin{itemize}
            \item Krümmungsverhalten
            \item Tangentensteigungen von speziellen Punkten
        \end{itemize}
    \item Graph der Funktion mit allen berechneten Werten
    \item Wertemenge der Funktion
\end{enumerate}
\end{merke}

\begin{bsp}{}{}
Betrachtet wird die Funktion $f(x)= \dfrac{1}{16} x^4 +\dfrac{1}{4} x^3$ mit $\mathds{D}_f =\mathds{D}_{max}$ dem maximalen Definitionsbereich. Es soll für die Funktion eine vollständige Kurvendiskussion durchgeführt werden.
\begin{enumerate}
    \item Bestimmung des maximalen Definitionsbereichs. $\mathds{D}_f = \mathds{R}$
    \item Untersuchungen der Symmetrie
    $$f(-x) = \frac{1}{16} \cdot (-x)^4 + \frac{1}{4} \cdot (-x)^3 = \frac{1}{16} \cdot x^4 - \frac{1}{4} \cdot x^3 \neq \pm f(x)$$ Damit ist der Graph $G_f$ weder punktsymmetrisch zum Ursprung noch achsensymmetrisch zur y-Achse.
    \item Verhalten an den Rändern des Definitionsbereichs
    $$\lim_{x\rightarrow \infty} f(x) = \lim_{x\rightarrow \infty} \dfrac{1}{16}x^4 +\dfrac{1}{4}x^3 = \infty$$
        $$\lim_{x\rightarrow -\infty} f(x) = \lim_{x\rightarrow -\infty} \dfrac{1}{16}x^4 +\dfrac{1}{4}x^3 = \infty$$
        Das Verhalten von Polynomen an den Rändern des Definitionsbereichs wird durch den Summanden mit der höchsten Potenz (Grad des Polynoms) festgelegt. In diesem Beispiel ist der Grad $n=4$ und damit streben die Funktionswerte an den Rändern des Definitionsbereichs gegen Unendlich.
    \item Gemeinsame Punkte mit den Koordinatenachsen
    \begin{itemize}
        \item Schnittpunkt mit der y-Achse
    
    \begin{equation*}
        f(0) = \dfrac{1}{16}0^4 +\dfrac{1}{4}0^3 = 0
    \end{equation*}
    Damit liegt der Schnittpunkt mit der y-Achse bei $SP_y(0|0)$.
    \item Schnittpunkte mit der x-Achse
    \begin{equation*}
        \begin{split}
         0 &= \dfrac{1}{16} x^3 (x+4)\\
             x_1 =x_2=x_3&= 0 \longrightarrow SP_{x_1} = SP_{x_2} = SP_{x_3}(0|0) \\
             x_4&= -4 \longrightarrow SP_{x_4}(-4|0)
        \end{split}
\end{equation*}
    \end{itemize}
  \item Monotonie des Graphen $G_f$ der Funktion $f$  
    \begin{itemize}
        \item Bildung der Ableitungen
    \begin{equation*}
        \begin{split}
        f(x) &= \dfrac{1}{16} x^4 +\dfrac{1}{4} x^3\\
            f'(x)&= \dfrac{1}{4} x^3 +\dfrac{3}{4} x^2\\
            f''(x) &= \dfrac{3}{4} x^2 + \dfrac{3}{2} x
        \end{split}
\end{equation*}
       \item Berechnung der Nullstellen der ersten Ableitung 
           \begin{equation*}
        \begin{split}
         0 &= \dfrac{1}{4} x^3 +\dfrac{3}{4} x^2\\
         0 &= \dfrac{1}{4} x^2 (x+3)\\
         x_1 =x_2 &= 0\\
         x_3&= -3
        \end{split}
\end{equation*}

Monotonietabelle:
\begin{center}\begin{tabular}{||c|c|c|c|c|c||}
    \hline
    $x$& $ -\infty <x<-3 $ & $ x = -3$ &$ -3<x<0 $ & $x=0 $& $ 0<x<\infty $\\
    \hline \hline
    $\frac{1}{4} x^2$ & + &  0 & + &  & +  \\
    \hline
    $(x +3)$ & - & 0 & + &  & + \\
    \hline

    $f'(x)$ & - & 0 & + & 0 & +\\ 
    \hline
    \hline
    $G_f$ & smf & $TiP(-3|-\frac{27}{16})$ & smw  & $TP(0|0) $& smw\\
    \hline
\end{tabular}
\end{center}
Berechnung der Koordinaten:
           \begin{equation*}
        \begin{split}
         f(-3)&= \dfrac{1}{16} (-3)^4 +\dfrac{1}{4} (-3)^3\\
         &= -\dfrac{27}{16}\approx -1,69\\
         f(0) &= \dfrac{1}{16} 0^4 +\dfrac{1}{4} 0^3 = 0
        \end{split}
\end{equation*}
Durch die Monotonietabelle ist es möglich, dass gleichzeitig die Art der Extrempunkte bestimmt werden kann.
    \end{itemize}
    \item Bestimmung des Krümmungsverhalten
\end{enumerate} 

\end{bsp}
\end{section}
\chapter{Analytische Geometrie}
\section{Das 3-dim Koordinatensystem}
blahblah

\chapter{Stochastik}
\chapter{Anhang}
\section{Polynomdivision}\index{Polynomdivision}
Die Polynomdivision wird eingesetzt um entweder den Grad eines Polynoms zu verringern oder die Bruchform einer gebrochenrationalen Funktion in die Summenform umzuwandeln.
\subsection{Polynomdivision bei Polynomen}\index{Polynomdivision!Polynome}\label{polynomdivision}

    Bei einem Polynom wird die Polynomdivision angewendet um den Grad des Polynoms zu verringern und die Faktorisierte Darstellung des Polynoms zu erzeugen. Als Beispiel wird hier die Funktion $f(x) = x^3-6x^2-x+6$ verwendet. Hierbei ist es Ziel, dass man das Polynom als Produkt der Nullstellen darstellen kann.\\
    Es gilt folgender Zusammenhang: $$f(x) = x^3-6x^2-x+6 =  \polyfactorize{x^3-6x^2-x+6}.$$ Aus dieser Darstellung folgt, dass die Funktion $f$ die Nullstellen $x_1 = -1, x_2 = 1$ und $x_3= 6$ hat. Um die Polynomdivision durchführen zu können, muss man als erstes eine Nullstelle kennen und dann das Polynom durch diese Nullstelle teilen. Nach der Division erhält man jeweils eine neues Polynom geringeren Grades. Dieses Polynom muss dann wieder mit der Polynomdivision vom Grad reduziert werden.
    \begin{bsp}{Die Polynomdivision bei Polynomen}{} 
Man rät die Nullstelle $x=1$ und führt jetzt die Polynomdivision durch:\\
 \polylongdiv{x^3- 6x^2 - x +6}{x-1}\\
 Das neue Polynom kann jetzt mit der Mitternachtsformel weiter bearbeitet werden oder erneut durch eine Polynomdivision bearbeitet werden. \\
  \polylongdiv{x^2-5x^1 -6}{x+1}\\
  oder
  $$ x_{1,2} = \dfrac{5\pm \sqrt{(-5)^2 - 4\cdot 1 \cdot (-6)}}{2\cdot 1} = \dfrac{5 \pm \sqrt{49} }{2} = \dfrac{5\pm 7}{2}$$
 $$ x_1 = -1$$
 $$ x_2= 6$$
\begin{center}
    \begin{tikzpicture}
        \begin{axis}[xmin= -3.5, xmax = 8.5, ymin= -35, ymax= 12,
        axis lines = middle, 
        xtick={-2, -1, 0, ..., 8},
        xlabel = $x$,
        ylabel=$y$]
            \addplot[color= red, samples = 300, domain= -2.5:6.5]{x^3-6*x^2-x+6};
        \end{axis}
    \end{tikzpicture}     
\end{center}
\end{bsp}
\begin{b8d}{Rest der Polynomdivision}{rest}
  Bei der Polynomdivision zur Bestimmung der Nullstellen eines Polynoms treten \textcolor{red}{keine Restterme} auf.   
\end{b8d}
\subsection{Polynomdivision bei gebrochenrationalen Funktionen}\index{Polynomdivision!gebrochenrationale Funktionen}\label{polynomdivisionBruch}
Die zweite Anwendung der Polynomdivision besteht in der Umwandlung der Bruchform in die Summenform der ganzrationalen Funktionen. Hierbei wird die Polynomdivision durchgeführt indem man den Zähler der Funktion durch seinen Nenner dividiert.
\begin{bsp}{Polynomdivision ganzrationaler Funktionen}{}
Gegeben ist die Funktion $$f(x)=\dfrac{x^2+x-2}{2x-4}$$ mit $\mathds{D}_f=\mathds{R}\setminus \left\{ 2 \right\}$. Durch die Polynomdivision soll die Bruchform in die Summenform überführt werden.\\
\polylongdiv{x^2+x-2}{2x-4}\\
Damit lautet die Summenform wie folgt:
$$f(x)=\dfrac{x^2+x-2}{2x-4} = \dfrac{1}{2} x +\dfrac{3}{2} +\dfrac{4}{2x-4} $$
\end{bsp}
\subsection{Symmetrie} \index{Symmetrie}
Die Untersuchung der Symmetrie erfolgt für Funktionen immer nach dem selben Muster. Es wird im allgemeinen Untersucht, ob entweder \begin{equation}\label{f(x)}
    f(-x) = f(x)
\end{equation} oder 
\begin{equation} \label{-f(x)}
f(-x) = -f(x) 
\end{equation} gilt. Für die Gleichung \ref{f(x)} ist der Graph der Funktion $f$ achsensymmetrisch zur y-Achse. Für die Gleichung \ref{-f(x)} ist der Graph der Funktion $f$ punktsymmetrisch zum Ursprung. Gilt keine der beiden Gleichungen kann keine Aussage zur Symmetrie zur y-Achse bzw. zum Ursprung getroffen werden. 
\begin{center}
    \begin{tikzpicture}
        \begin{axis}[xmin= -4.1, xmax = 4, ymin= -15.1, ymax= 20.5,
        axis lines = middle, 
        ymajorgrids=true,
        xmajorgrids=true,
        xtick={-4, -3,  ..., 2, 3, 4},
        ytick={-15, -10, -5, ..., 20},
        xlabel = $x$,
        ylabel=$y$]
            \addplot[color = red, samples = 300, domain= -3.7:3.7]{x^2};
         \draw (1.5,15)   node [right] {$f(x) = x^3$};  
            \addplot[color= black, samples = 300, domain= -3.7:3.7]{x^3};
            \draw (-3,12)   node [right] {\textcolor{red}{$g(x) = x^2$}}; 
            
        \end{axis}
    \end{tikzpicture}  
\end{center}
\subsection{Übersicht von Ableitungsregeln} \index{Ableitung! Übersicht}

\printindex

\end{document}