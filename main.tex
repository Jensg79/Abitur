\documentclass[a4paper]{report}
\usepackage[utf8]{inputenc}
\usepackage[ngerman]{babel}
\usepackage{amsmath, amsfonts, amsthm, graphicx, geometry, lipsum}
\usepackage{hyperref}
\usepackage{xcolor}
\usepackage{polynom}
\polyset{%
    vars=x,
    style=C,
    div=:,
    delims={(}{)}}
\usepackage{dsfont}
\colorlet{shadecolor}{yellow!15}

\usepackage{thmtools}
\hypersetup{colorlinks =true, linkcolor = blue, urlcolor =red, pdftitle={Mathematik in Q11}}
\usepackage{fancyvrb}
\usepackage{fancyhdr, lastpage}
\pagestyle{fancy}
\usepackage{xcolor}
\lhead{J. Gundelwein}
\rhead{Gymnasium Gröbenzell}
\cfoot{Seite \thepage \ von \pageref{LastPage}}
\usepackage[Glenn]{fncychap}
%Options: Sonny, Glenn, Lenny, Conny, Rejne, Bjarne, Bjornstrup
\usepackage{xcolor}
\usepackage{afterpage}
%\usepackage{tikz}

\usepackage[most]{tcolorbox}
\usepackage{pgfplots}
\pgfplotsset{compat=1.15, width =10cm}
\usepackage{mathrsfs}
\usetikzlibrary{arrows}

\newtcbtheorem[number within=section]{defi}
    {Definition}{colback=red!5,colframe=red!35!black,fonttitle=\bfseries,title after break={Definition (Fortsetzung)}}{theorem}
\newtcbtheorem[number within=section]{bsp}
    {Beispiel}{colback=white,colframe=blue!70!green,fonttitle=\bfseries, breakable,title after break={Beispiel (Fortsetzung)}}{theorem}
\newtcbtheorem[number within=section]{merke}{Merke} {colback=green!5,colframe=green!35!black,fonttitle=\bfseries, breakable,title after break={Merke (Fortsetzung)}}{th}
\newtcbtheorem[number within=section]{bem}
    {Bemerkung}{colback=red!5,colframe=red!35!black,breakable, fonttitle=\bfseries,title after break={Bemerkung (Fortsetzung)}}{theorem}
\usepackage{setspace}
\usepackage{makeidx}
\renewcommand{\indexname}{Sachregister}
\makeindex
\begin{document}

\onehalfspacing
\begin{titlepage}

\begin{center}
%\renewcommand{\baselinestretch}{1}  % Zeilenabstand einfach
\huge \textbf{Gymnasium Gröbenzell}
\vspace{0.2cm}\\
\Large \textbf{ Mathematik Q11}
\vspace{1cm}\\
\LARGE \textbf{Abiturvorbereitung Mathematik}
\vspace{1cm}\\
\end{center}
\end{titlepage}
\tableofcontents
\newpage
\chapter{Analysis}
\begin{section}{Lokales Differenzieren}\index{Differenzieren!lokal}
\begin{subsection}{Die mittlere Änderungsrate}
\begin{defi}{Die mittlere Änderungsrate}{maer}\index{Änderungsrate!mittlere}
Der Quotient $$\dfrac{\Delta y}{\Delta x} = \dfrac{f(b) - f(a)}{b - a}$$ heißt mittler Änderungsrate oder Differenzenquotient der Funktion f im Intervall $\left[a;b\right]$.\\
Geometrisch gedeutet ist der Differenzenquotient \index{Änderungsrate!Differenzenquotient} die Steigung der Sekante durch die Punkte $P(a|f(a))$ und $Q(b|f(b))$.
\end{defi}
\begin{bsp}{Ein Beispiel für die mittlere Änderungsrate}{bspmaer}
Gegeben sind die Punkte $P(3|4)$ und $Q(6|1)$ und damit die mittlere Änderungsrate: $$ \dfrac{\Delta y}{\Delta x} = \dfrac{4-1}{3-6} = \dfrac{3}{-3} = -1$$
\end{bsp}
\end{subsection}

\subsection{Definition der Ableitung}{abl}
Übergang von der Sekante zur Tangente:\\
Wir betrachten die Normalparabel $$ y=x^2 $$ und darauf den Punkt $P(1,5|2,25)$. Zur Bestimmung des Anstiegs bildet man den Differenzenquotienten zwischen P und einen beliebigem Punkt $Q(x|x^2)$. Bildet man für die beliebigen Werte von Q den Differenzenquotienten, so ergibt sich folgende Gleichung: 
\begin{bsp*}{Differenzenquotient für $f(x) =x^2 $}\index{Änderungsrate!Differenzenquotient!Beispiel}
  
  $$ m_s(x)= \dfrac{\Delta y}{\Delta x} = \dfrac{f(x) - 2,25}{x - 1,5} = \dfrac{x^2 - 2,25}{x- 1,5} = \dfrac{(x-1,5)(x+1,5)}{x-1,5} =x+1,5 $$

\end{bsp*}
Nähert sich der Punkt Q beliebig nahe an P an, so unterscheidet sich die Steigung der Sekante immer weniger von 3. Man spricht in diesem Fall auch von einem Grenzwert.\
\begin{merke*}{Wichtige Informationen zur Definition  der Ableitung}
\index{Ableitung!Definition}
\begin{itemize}
 \item Nähern sich die Funktionswerte $f(x)$ einer Funktion $f$ einer Zahl $a$ beliebig genau an, das ist der Fall wenn sich die Werte für $x$ von links und von rechts an $x_0$ beliebig genau annnähern, so heißt $a$ Grenzwert der Funktion $f(x)$. $\lim_{x \rightarrow x_0}$ 
 \item Der Punkt Q laufe auf einer Kurve von links und rechts auf den Punkt P zu. Ergibt sich dabei ein gemeinsamer Grenzwert der Sekante, heißt die Grenzgerade Tangente an die Kurve im Punkt P.
 \item Die Steigung eines Graphen im Punkt P ist gleich der Steigung der Tangente in diesem Punkt.
\end{itemize}
\end{merke*}
\begin{merke*}{Die h-Methode zur Berechnung der Steigung der Tangente}{hmethode}\index{Ableitung!h-Methode}
Wir führen eine Hilfsvariable h ein, diese gibt die Abweichung der Koordinate $x$ von der Koordinate $x_0$ angibt. Damit ist $x=x_0 + h$. Mit dieser Überlegung ist es jetzt möglich den Grenzwert zu bestimmen: $$m_t =\lim_{h\rightarrow 0} m_t(x_0 +h) =\lim_{h\rightarrow 0} \dfrac{f(x_0 + h) - f(x_0)}{h}.$$
\end{merke*}
\subsection{Differenzierbarkeit und Lineare Transformation}\index{Differenzieren!Differenzierbarkeit} \index{Ableitung!Differenzierbarkeit}
\begin{defi}{Die Differenzierbarkeit einer Funktion}
 Eine Funktion f heißt an der Stelle $x_0$ im Inneren des Definitionsbereichs $D_f$ differenzierbar, wenn der Grenzwert $f'(x) =lim_{h\rightarrow 0}\dfrac{(f(x_0 +h) -f(x_0)}{h}$ existiert, sich also bei links- und rechtsseitiger Annäherung der gleiche Wert ergibt. Der Grenzwert des Differenzenquotient heißt Differentialquotient.
\end{defi}
\begin{bsp*}{Die Überprüfung der Differenzierbarkeit mit der h-Methode}\index{Differenzieren!Differenzierbarkeit!Beispiel}
    Gegeben ist die Funktion $f(x)= x^2 + |x|$, wir überprüfen die Ableitung an der Stelle $P(0|f(0))$.

$$f(x)= \begin{cases}
              x^2 + x & \text{für}\hspace{0.5cm} x\geq 0\\
              x^2 - x & \text{für}\hspace{0.5cm}  x< 0 
         \end{cases}$$
Für die Untersuchung der Differenzierbarkeit sind folgende Grenzwerte zu überprüfen::
\begin{equation*}
\begin{split}
f'(0) &= \lim_{h\rightarrow0^-} \dfrac{f(0+h)-f(0)}{h}\\ &=\lim_{h\rightarrow 0^-} \dfrac{(0+h)^2 - (0+h) -0}{h}\\ &= \lim_{h\rightarrow 0^-} \dfrac{h^2-h}{h} = \lim_{h\rightarrow 0^-} h-1=-1 
\end{split}
\end{equation*}
\begin{equation*}
\begin{split}
f'(0)&=\lim_{h\rightarrow 0^+}\dfrac{f(0+h)-f(0)}{h}\\
&=\lim_{h\rightarrow 0^+} \dfrac{(0+h)^2 + (0+h) -0}{h}\\ 
&=\lim_{h\rightarrow 0^+} \dfrac{h^2 + h}{h} = \lim_{h\rightarrow 0^+} h+1 =1
\end{split}
\end{equation*}
Der Differenzenquotient hat daher für $x\rightarrow 0$ keinen Grenzwert, die Funktion $f$ ist damit an der Stelle $x_0 = 0$ nicht differenzierbar. Der Graph $G_f$ hat an dieser Stelle einen "`Knick"'.
\end{bsp*}

\begin{bem*}{}\index{Differenzieren!Differenzierbarkeit!Lineare Transformationen}\index{Lineare Transformation}
Die Lineare Transformation der Betragsfunktion ändert nichts an der Nicht-Differenzierbarkeit der Funktion.
\end{bem*}
Während der letzten Schulajahre wurden verschiedene Funktionsarten einer linearen Transformation unterzogen. Als Beispiel wird hier auf die Funktionen: 
 \begin{equation*}
    \begin{split}
        f(x) &= x^2\\
        g(x) &= \sin{(x)}\\
        h(x) &= \cos{(x)}
    \end{split}
\end{equation*}
verwiesen.\\ 
In Jahrgangsstufe 9 wurde die Normalparabel auf unterschiedlichste Arten transformiert. In der 10. Jahrgangsstufe wurden Trigonometrischen Funktionen transformiert.
\begin{bsp*}{Transformationen am Bespiel der Normalparabel}{}\index{Lineare Transformation!Beispiel}
Der Graph der Funktion $f(x) = x^2$ läßt sich wie folgt transformieren:
\begin{itemize}
\item $g_1(x) = -x^2$ $\Longrightarrow$ Normalparabel an der $x-$Achse gespiegelt
\item $g_2(x) =   x^2 +a$ $\Longrightarrow$ $G_f$  Richtung der y--Achse um a verschoben
\item $g_3(x) =(x-a)^2$ $\Longrightarrow$ $G_f$ in Richtung der x--Achse um a verschoben
\item $g_4(x) =a\cdot x^2$ und $a>0$, $\Longrightarrow$ $G_f$ in Richtung der y--Achse mit dem Faktor a gestreckt bzw. gestaucht
\end{itemize}
\end{bsp*}
\begin{merke}{Allgemeine Darstellung der Linearen Transformationen} 
Die lineare Transformation lässt sich allgemein wie folgt schreiben:\index{Lineare Transformation!allgemein}\\
Wir erhalten aus dem Graphen $G_f$ der Funktion f den Graphen der Funktion g mit:
\begin{itemize}
\item $g(x) = - f(x)$, indem man $G_f$ an der x--Achse spiegelt
\item $g(x) =f(-x)$, indem man $G_f$ an der y--Achse spiegelt
\item $g(x) = f(x) +a$ indem man $G_f$ in Richtung der y--Achse um a verschiebt
\item $g(x) =f(x-a)$ indem man $G_f$ in Richtung der x--Achse um a verschiebt
\item $g(x) =a\cdot f(x)$ und $a>0$, indem man $G_f$ in Richtung der y--Achse mit dem Faktor a streckt bzw. staucht
\item $g(x) =f(a\cdot x)$ und  $a>0$, indem man $G_f$ in Richtung der x--Achse mit dem Faktor $\dfrac{1}{a}$ staucht bzw. streckt 
\end{itemize}
\end{merke}
\end{section}

\begin{section}{Globales Differenzieren}\index{Differenzieren!global}
\subsection{Die Ableitungsfunktion}
\begin{defi}{Die Ableitungsfunktion}{ablf}
    \index{Ableitung!Ableitungsfunktion}
    Zu einer Funktion $f$ heißt die Funktion $f': x \mapsto f'(x)$ die Ableitungsfunktion oder kurz die Ableitung von $f$. Die Ableitung einer Funktion $f$ ist diejenige Funktion, die jeder Stelle $x$, an der $f$ differenzierbar ist, die Ableitung $f'(x)$ zuordnet:
$$f'(x)=\lim_{h\rightarrow 0} \dfrac{f(x + h) - f(x)}{h}.$$ 
\end{defi}
\begin{bem}{Bedeutung der Ableitungsfunktion}{bedeutungAblf}
   Die Ableitungsfunktion $f'$ beschreibt die Steigung der Funktion $f$ an jeder Stelle $x\in \mathds{R}$. \\
   Bei der Ableitung von Polynomen verringert sich der Grad der Funktion bei jeder Ableitung um 1.\\
   
\end{bem}
\begin{bsp}{Beispielfunktion}{}\index{Ableitung!Ableitungsfunktion!Beispiel}
Die Funktion $f(x) = x^3 + x^2 -2x+1$ mit $x\in\mathds{R}$ ist ein Polynom vom Grad 3, die Funktion $f'(x) = 3x^2+2x-2$ ist die erste Ableitung der Funktion $f(x)$. Die Graphen der beiden Funktionen ist unten gegeben. 
\definecolor{ccqqqq}{rgb}{0.8,0,0}
\begin{center}
\begin{tikzpicture}[line cap=round,line join=round,>=triangle 45,x=1cm,y=1cm]
\begin{axis}[
x=1cm,y=1cm,
axis lines=middle,
ymajorgrids=true,
xmajorgrids=true,
xmin=-4.6725325047810795,
xmax=3.4865844816315783,
ymin=-3.970650919871985,
ymax=4.84273488148887,
xtick={-4,-3,...,3},
ytick={-3,-2,...,4},]
\clip(-4.6725325047810795,-3.970650919871985) rectangle (3.4865844816315783,4.84273488148887);
\draw[line width=1.2pt,smooth,samples=100,domain=-4.6725325047810795:3.4865844816315783] plot(\x,{(\x)^(3)+(\x)^(2)-2*(\x)+1});
\draw[line width=1.2pt,color=ccqqqq,smooth,samples=100,domain=-4.6725325047810795:3.4865844816315783] plot(\x,{3*(\x)^(2)+2*(\x)-2});
\begin{scriptsize}
\draw[color=black] (-2,-2.7) node {$f(x)$};
\draw[color=ccqqqq] (-2.2,3.5) node {$f'(x)$};
\end{scriptsize}
\end{axis}
\end{tikzpicture}
\end{center}
\end{bsp}
\subsection{Ableitung ganzrationaler Funktionen}
Die Ableitung von Funktionen mit der h-Methode ist für jede Funktion möglich. Allerdings wird das Ableiten damit schnell mühsam. Durch die Anwendung von Regeln zur Ableitung ganzrationaler Funktionen wird das Bilden der ersten Ableitung um einiges einfacher.\\
\begin{merke*}{Ableitungsregel Potenzfunktionen}{}\index{Ableitung!Regel!Potenzfunktion}
Die Potenzfunktion $f$ mit $f(x) = x^n$ und natürlichen Exponenten $n$ ist differenzierbar und es gilt: $$f'(x) = n\cdot x^{n-1}$$
\underline{Merkregel:} Der Term $x^n$ wird differenziert, indem man den Exponenten $n$ als Faktor voranstellt und den Exponenten um 1 verringert.
\end{merke*}
\begin{bsp}{Beispiele für das Ableiten von Potenzfunktionen}{}
\begin{itemize}
    \item $f(x)= x^3\ \longrightarrow f'(x) = 3x^2$
    \item $g(x) = \dfrac{3}{4} x^5 \ \longrightarrow g'(x)=  5\cdot \dfrac{3}{4} x^4 = \dfrac{15}{4}x^4$ 
    \item $h(x) = x^{\pi} \ \longrightarrow h'(x) = \pi \cdot x^{\pi -1}$
\end{itemize}
\end{bsp}
\subsection{Anwendung der Ableitungsfunktion - Newton-Verfahren}\index{Newton-Verfahren}
Innerhalb der Analysis ist die Bestimmung der Nullstellen eine der zentralen Aufgaben. Diese Berechnung erfolgt allerdings nur bei einer begrenzten Anzahl von Funktionsklassen problemlos. So gibt es bei Polynomen vom Grad $n\geq 4$ keine Lösungsformel\footnote{Für den Grad $n=2$ ist die Lösungsformel die bereits bekannte Mitternachtsformel: $x_{1,2} = \dfrac{-b\pm \sqrt{b^2 -4ac}}{2\cdot a}$} mehr. Für den Grad $n\geq 4$ gibt es bei Polynomen dann verschiedene Wege die Nullstellen zu bestimmen. Eine dieser Möglichkeiten ist die Polynomdivision\footnote{Die Polynomdivision ist nicht Teil des Lehrplans der Klasse 11, allerdings aus der 10. Klasse bekannt.}
\end{section}

\chapter{Analytische Geometrie}
\section{Das 3-dim Koordinatensystem}
blahblah

\chapter{Stochastik}
\chapter{Anhang}
\section{Polynomdivision}\index{Polynomdivision}
Die Polynomdivision wird eingesetzt um entweder den Grad eines Polynoms zu verringern oder die Bruchform einer gebrochenrationalen Funktion in die Summenform umzuwandeln.
\subsection{Polynomdivision bei Polynomen}\index{Polynomdivision!Polynome}
\begin{bsp}{Die Polynomdivision}
    Bei einem Polynom wird die Polynomdivision angewendet um den Grad des Polynoms zu verringern und die Faktorisierte Darstellung des Polynoms zu erzeugen. Als Beispiel wird hier die Funktion $f(x) = x^3-6x^2-x+6$ verwendet. Hierbei ist es Ziel, dass man das Polynom als Produkt der Nullstellen darstellen kann.
    \begin{tikzpicture}
        \begin{axis}[xmin= -3.5, xmax = 8.5, ymin= -35, ymax= 12,
        axis lines = middle, 
        xtick={-2, -1, 0, ..., 8},
        xlabel = $x$,
        ylabel=$y$]
            \addplot[color= red, samples = 300, domain= -2.5:6.5]{x^3-6*x^2-x+6};
        \end{axis}
    \end{tikzpicture}
\end{bsp}
\begin{center}
  \polylongdiv{x^3- 6x^2 - x +6}{x-1}
\end{center}  
\begin{center}
  \polyfactorize{x^3-6x^2-x+6}
\end{center} 

\printindex

\end{document}