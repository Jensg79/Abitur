\documentclass[a4paper]{report}
\usepackage[utf8]{inputenc}
\usepackage[ngerman]{babel}
\usepackage{amsmath, amsfonts, amsthm, graphicx, geometry, lipsum}
\usepackage{hyperref}
\usepackage{xcolor}

\colorlet{shadecolor}{yellow!15}

\usepackage{thmtools}
\hypersetup{colorlinks =true, linkcolor = blue, urlcolor =red, pdftitle={Mathematik in Q11}}
\usepackage{fancyvrb}
\usepackage{fancyhdr, lastpage}
\pagestyle{fancy}
\usepackage{xcolor}
\lhead{J. Gundelwein}
\rhead{Gymnasium Gröbenzell}
\cfoot{Seite \thepage \ von \pageref{LastPage}}
\usepackage[Glenn]{fncychap}
%Options: Sonny, Glenn, Lenny, Conny, Rejne, Bjarne, Bjornstrup
\usepackage{xcolor}
\usepackage{afterpage}
\usepackage{tikz} 
\usepackage[most]{tcolorbox}


\newtcbtheorem[number within=section]{defi}
    {Definition}{colback=red!5,colframe=red!35!black,fonttitle=\bfseries,title after break={Definition (Fortsetzung)}}{theorem}
\newtcbtheorem[number within=section]{bsp}
    {Beispiel}{colback=white,colframe=blue!70!green,fonttitle=\bfseries, breakable,title after break={Beispiel (Fortsetzung)}}{theorem}
\newtcbtheorem[number within=section]{merke}{Merke} {colback=green!5,colframe=green!35!black,fonttitle=\bfseries, breakable,title after break={Merke (Fortsetzung)}}{th}
\newtcbtheorem[number within=section]{bem}
    {Bemerkung}{colback=red!5,colframe=red!35!black,breakable, fonttitle=\bfseries,title after break={Bemerkung (Fortsetzung)}}{theorem}
\usepackage{setspace}

\usepackage{makeidx}
\renewcommand{\indexname}{Sachregister}
\makeindex
\begin{document}

\onehalfspacing
\begin{titlepage}

\begin{center}
%\renewcommand{\baselinestretch}{1}  % Zeilenabstand einfach
\huge \textbf{Gymnasium Gröbenzell}
\vspace{0.2cm}\\
\Large \textbf{ Mathematik Q11}
\vspace{1cm}\\
\LARGE \textbf{Abiturvorbereitung Mathematik}
\vspace{1cm}\\
\end{center}
\end{titlepage}
\tableofcontents
\newpage
\chapter{Analysis}
\begin{section}{Lokales Differenzieren}\index{Differenzieren!lokal}
\begin{subsection}{Die mittlere Änderungsrate}
\begin{defi}{Die mittlere Änderungsrate}{maer}\index{Änderungsrate!mittlere}
Der Quotient $$\dfrac{\Delta y}{\Delta x} = \dfrac{f(b) - f(a)}{b - a}$$ heißt mittler Änderungsrate oder Differenzenquotient der Funktion f im Intervall $\left[a;b\right]$.\\
Geometrisch gedeutet ist der Differenzenquotient \index{Änderungsrate!Differenzenquotient} die Steigung der Sekante durch die Punkte $P(a|f(a))$ und $Q(b|f(b))$.
\end{defi}
\begin{bsp}{Ein Beispiel für die mittlere Änderungsrate}{bspmaer}
Gegeben sind die Punkte $P(3|4)$ und $Q(6|1)$ und damit die mittlere Änderungsrate: $$ \dfrac{\Delta y}{\Delta x} = \dfrac{4-1}{3-6} = \dfrac{3}{-3} = -1$$
\end{bsp}
\end{subsection}

\subsection{Definition der Ableitung}{abl}
Übergang von der Sekante zur Tangente:\\
Wir betrachten die Normalparabel $$ y=x^2 $$ und darauf den Punkt $P(1,5|2,25)$. Zur Bestimmung des Anstiegs bildet man den Differenzenquotienten zwischen P und einen beliebigem Punkt $Q(x|x^2)$. Bildet man für die beliebigen Werte von Q den Differenzenquotienten, so ergibt sich folgende Gleichung: 
\begin{bsp*}{Differenzenquotient für $f(x) =x^2 $}\index{Änderungsrate!Differenzenquotient!Beispiel}
  
  $$ m_s(x)= \dfrac{\Delta y}{\Delta x} = \dfrac{f(x) - 2,25}{x - 1,5} = \dfrac{x^2 - 2,25}{x- 1,5} = \dfrac{(x-1,5)(x+1,5)}{x-1,5} =x+1,5 $$

\end{bsp*}
Nähert sich der Punkt Q beliebig nahe an P an, so unterscheidet sich die Steigung der Sekante immer weniger von 3. Man spricht in diesem Fall auch von einem Grenzwert.\
\begin{merke*}{Wichtige Informationen zur Definition  der Ableitung}\ref{defabl}
\index{Ableitung!Definition}
\begin{itemize}
 \item Nähern sich die Funktionswerte $f(x)$ einer Funktion $f$ einer Zahl $a$ beliebig genau an, das ist der Fall wenn sich die Werte für $x$ von links und von rechts an $x_0$ beliebig genau annnähern, so heißt $a$ Grenzwert der Funktion $f(x)$. $\lim_{x \rightarrow x_0}$ 
 \item Der Punkt Q laufe auf einer Kurve von links und rechts auf den Punkt P zu. Ergibt sich dabei ein gemeinsamer Grenzwert der Sekante, heißt die Grenzgerade Tangente an die Kurve im Punkt P.
 \item Die Steigung eines Graphen im Punkt P ist gleich der Steigung der Tangente in diesem Punkt.
\end{itemize}
\end{merke*}
\begin{merke*}{Die h-Methode zur Berechnung der Steigung der Tangente}{hmethode}\index{Ableitung!h-Methode}
Wir führen eine Hilfsvariable h ein, diese gibt die Abweichung der Koordinate $x$ von der Koordinate $x_0$ angibt. Damit ist $x=x_0 + h$. Mit dieser Überlegung ist es jetzt möglich den Grenzwert zu bestimmen: $$m_t =\lim_{h\rightarrow 0} m_t(x_0 +h) =\lim_{h\rightarrow 0} \dfrac{f(x_0 + h) - f(x_0)}{h}.$$
\end{merke*}
\subsection{Differenzierbarkeit und Lineare Transformation}\index{Differenzieren!Differenzierbarkeit} \index{Ableitung!Differenzierbarkeit}
\begin{defi}{Die Differenzierbarkeit einer Funktion}
 Eine Funktion f heißt an der Stelle $x_0$ im Inneren des Definitionsbereichs $D_f$ differenzierbar, wenn der Grenzwert $f'(x) =lim_{h\rightarrow 0}\dfrac{(f(x_0 +h) -f(x_0)}{h}$ existiert, sich also bei links- und rechtsseitiger Annäherung der gleiche Wert ergibt. Der Grenzwert des Differenzenquotient heißt Differentialquotient.
\end{defi}
\begin{bsp*}{Die Überprüfung der Differenzierbarkeit mit der h-Methode}\index{Differenzieren!Differenzierbarkeit!Beispiel}
    Gegeben ist die Funktion $f(x)= x^2 + |x|$, wir überprüfen die Ableitung an der Stelle $P(0|f(0))$.

$$f(x)= \begin{cases}
              x^2 + x & \text{für}\hspace{0.5cm} x\geq 0\\
              x^2 - x & \text{für}\hspace{0.5cm}  x< 0 
         \end{cases}$$
Für die Untersuchung der Differenzierbarkeit sind folgende Grenzwerte zu überprüfen::
\begin{equation*}
\begin{split}
f'(0) &= \lim_{h\rightarrow0^-} \dfrac{f(0+h)-f(0)}{h}\\ &=\lim_{h\rightarrow 0^-} \dfrac{(0+h)^2 - (0+h) -0}{h}\\ &= \lim_{h\rightarrow 0^-} \dfrac{h^2-h}{h} = \lim_{h\rightarrow 0^-} h-1=-1 
\end{split}
\end{equation*}
\begin{equation*}
\begin{split}
f'(0)&=\lim_{h\rightarrow 0^+}\dfrac{f(0+h)-f(0)}{h}\\
&=\lim_{h\rightarrow 0^+} \dfrac{(0+h)^2 + (0+h) -0}{h}\\ 
&=\lim_{h\rightarrow 0^+} \dfrac{h^2 + h}{h} = \lim_{h\rightarrow 0^+} h+1 =1
\end{split}
\end{equation*}
Der Differenzenquotient hat daher für $x\rightarrow 0$ keinen Grenzwert, die Funktion $f$ ist damit an der Stelle $x_0 = 0$ nicht differenzierbar. Der Graph $G_f$ hat an dieser Stelle einen "`Knick"'.
\end{bsp*}

\begin{bem*}{}\index{Differenzieren!Differenzierbarkeit!Lineare Transformationen}\index{Lineare Transformation}
Die Lineare Transformation der Betragsfunktion ändert nichts an der Nicht-Differenzierbarkeit der Funktion.
\end{bem*}
Während der letzten Schulajahre wurden verschiedene Funktionsarten einer linearen Transformation unterzogen. Als Beispiel wird hier auf die Funktionen: 
 \begin{equation*}
    \begin{split}
        f(x) &= x^2\\
        g(x) &= \sin{(x)}\\
        h(x) &= \cos{(x)}
    \end{split}
\end{equation*}
verwiesen.\\ 
In Jahrgangsstufe 9 wurde die Normalparabel auf unterschiedlichste Arten transformiert. In der 10. Jahrgangsstufe wurden Trigonometrischen Funktionen transformiert.
\begin{bsp*}{Transformationen am Bespiel der Normalparabel}\index{Lineare Transformation!Beispiel}
Der Graph der Funktion $f(x) = x^2$ läßt sich wie folgt transformieren:
\begin{itemize}
\item $g_1(x) = -x^2$ $\Longrightarrow$ Normalparabel an der $x-$Achse gespiegelt
\item $g_2(x) =   x^2 +a$ $\Longrightarrow$ $G_f$  Richtung der y--Achse um a verschoben
\item $g_3(x) =(x-a)^2$ $\Longrightarrow$ $G_f$ in Richtung der x--Achse um a verschoben
\item $g_4(x) =a\cdot x^2$ und $a>0$, $\Longrightarrow$ $G_f$ in Richtung der y--Achse mit dem Faktor a gestreckt bzw. gestaucht
\end{itemize}
\end{bsp*}
\begin{merke}{Allgemeine Darstellung der Linearen Transformationen} 
Die lineare Transformation lässt sich allgemein wie folgt schreiben:\index{Lineare Transformation!allgemein}\\
Wir erhalten aus dem Graphen $G_f$ der Funktion f den Graphen der Funktion g mit:
\begin{itemize}
\item $g(x) = - f(x)$, indem man $G_f$ an der x--Achse spiegelt
\item $g(x) =f(-x)$, indem man $G_f$ an der y--Achse spiegelt
\item $g(x) = f(x) +a$ indem man $G_f$ in Richtung der y--Achse um a verschiebt
\item $g(x) =f(x-a)$ indem man $G_f$ in Richtung der x--Achse um a verschiebt
\item $g(x) =a\cdot f(x)$ und $a>0$, indem man $G_f$ in Richtung der y--Achse mit dem Faktor a streckt bzw. staucht
\item $g(x) =f(a\cdot x)$ und  $a>0$, indem man $G_f$ in Richtung der x--Achse mit dem Faktor $\dfrac{1}{a}$ staucht bzw. streckt 
\end{itemize}
\end{merke}
\end{section}

\begin{section}{Globales Differenzieren}
\subsection{Die Ableitungsfunktion}
\begin{defi}{Die Ableitungsfunktion}{ablf}
    \index{Ableitung!Ableitungsfunktion}
    Zu einer Funktion $f$ heißt die Funktion $f': x \mapsto f'(x)$ die Ableitungsfunktion oder kurz die Ableitung von $f$. Die Ableitung einer Funktion $f$ ist diejenige Funktion, die jeder Stelle $x$, an der $f$ differenzierbar ist, die Ableitung $f'(x)$ zuordnet:
$$f'(x)=\lim_{h\rightarrow 0} \dfrac{f(x + h) - f(x)}{h}.$$ 

\end{defi}

\end{section}
\chapter{Analytische Geometrie}
\section{Das 3-dim Koordinatensystem}
\chapter{Stochastik}
\printindex

\end{document}