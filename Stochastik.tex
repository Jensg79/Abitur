\section{Wiederholung der Wahrscheinlichkeitsbegriffe}
\begin{merke}{Grundbegriffe der Wahrscheinlichkeit}{}
    Ein Ereignis A ist eine Teilmenge der Ergebnisraumes $\Omega$.\\
    Ein Ereignis A tritt ein, wenn das Versuchsergebnis $\omega$ in A enthalten ist, also wenn gilt: $\omega \in A$.\\
    Alle Elemente von $\Omega$, die nicht zum Ereignis A gehören, fasst man unter dem Namen Gegenereignis $\Bar{A}$ von $A$ zusammen. Damit ist $\Bar{A} = \Omega \setminus A$.
\end{merke}
\begin{b8d}{Mengenalgebra}{}
    \begin{itemize}
        \item Schnittmenge $A\cap B$: Menge der gemeinsamen Ergebnisse von A und B
        \item Vereinigungsmenge $A\cup B$: Menge der Ergebnisse die sowohl in A als auch in B liegen.
    \end{itemize}
\end{b8d}
\begin{merke}{Unvereinbar}{}
Zwei Ereignisse A und B heißen unvereinbar, wenn $A\cap B = \{ \}$.
\end{merke}
\section{Axiomatischer Aufbau der Wahrscheinlichkeit}
Der axiomatische Aufbau der Wahrscheinlichkeitsrechnung geht auf den russischen Mathematiker Kolmogoroff zurück. Hierbei wird versucht, die gesamte Stochastik mit möglichst wenigen Annahmen zu beschreiben. Die sogenannten Axiome sind hierbei Grundannahmen die weder bewiesen noch widerlegt werden können. Hierbei wird die Berechnung der Wahrscheinlichkeit durch eine Funktion $P$ beschrieben.\\
\begin{defi}{}{Grundaxiome der Stochastik}
    Eine Funktion $P$, die jedem Ereignis A eine Wahrscheinlichkeit $P(A)$ zuordnet nennt man Wahrscheinlichkeitsverteilung genau dann wenn sie folgende Eigenschaften erfüllt.
\end{defi}
\section{Bedingte Wahrscheinlichkeit}\index{Wahrscheinlichkeit!bedingte Wahrscheinlichkeit}
Wahrscheinlichkeiten von Ereignissen können sich verändern, wenn bereits andere Ereignisse eingetreten sind. Um diesen Einfluss zu untersuchen, wird der Begriff der bedingten Wahrscheinlichkeit eingeführt.
\begin{defi}{Die bedingte Wahrscheinlichkeit}{}
\index{Bedingte Wahrscheinlichkeit}
   Die Wahrscheinlichkeit eines Ereignisses B unter der Bedingung, dass das Ereignis A bereits eingetreten ist lässt sich durch folgende Zusammenhang berechnen: $$P_A(B) = \dfrac{P(A\cap B)}{P(A)}$$   
\end{defi}
Durch die Betrachtung eines zweistufigen Baumdiagramms ist es möglich, die Definition für die bedingte Wahrsheinlichkeit zu zeigen.
\begin{merke*}{Die Herleitung der bedingten Wahrscheinlichkeit}{}
\begin{multicols}{2}
\scalebox{0.65}{
% allgemeines Layout des Baums
\tikzstyle{level 1}=[level distance=2.5cm, sibling distance=5cm]
\tikzstyle{level 2}=[level distance=2.5cm, sibling distance=4cm]
% definiert Knoten- und Endpunkte
% text width ändert die Boxbreite wobei 1em einem Zeichen entspricht
\tikzstyle{bag} = [circle, draw, text width=1em, inner sep=2pt, text centered]
\tikzstyle{end} = [circle, minimum width=4pt, fill, inner sep=0pt]
\begin{tikzpicture}[grow=down]
  \node[bag]{}
  child
  {
    node[bag] {$A$} % beschriftet Knoten 2 in erster Ebene mit X
    child
    {
      node (B) [bag] {$B$} % beschriftet Knoten 4 in zweiter Ebene mit D
      node [below of = B] {$P(A \cap B)$}
      edge from parent
      node[left]  {$P_{A}(B)$} % beschriftet Verbindung zu Knoten 4 (Ebene 2) mit p
} child {
      node[bag] {$\Bar{B}$} % beschriftet Knoten 3 in zweiter Ebene mit C
      edge from parent
      node[right]  {$P_{A}(\Bar{B})$} % beschriftet Verbindung zu Knoten 3 (Ebene 2) mit m
    }
    edge from parent
    node[left]  {$P(A)$} % beschriftet Verbindung zu Knoten 2 (Ebene 1) mit f
} child {
    node[bag] {$\Bar{A}$} % beschriftet Knoten 1 in erster Ebene mit A
    child
    {
      node[bag] {$B$} % beschriftet Knoten 2 in zweiter Ebene mit B
      edge from parent
      node[left]  {$P_{\Bar{A}}(B)$} % beschriftet Verbindung zu Knoten 2 (Ebene 2) mit k
} child {
      node[bag] {$\Bar{B}$} % beschriftet Knoten 1 in zweiter Ebene mit A
      edge from parent
      node[right]  {$P_{\Bar{A}}(B)$} % beschriftet Verbindung zu Knoten 1 (Ebene 2) mit i
    }
    edge from parent
    node[right]  {$P(A)$} % beschriftet Verbindung zu Knoten 1 (Ebene 1) mit w
  };
\end{tikzpicture}}

Nach der ersten Pfadregel berechnet sich die Wahrscheinlichkeit $P(A\cap B)$ durch das Produkt der einzelnen Pfadwahrscheinlichkeiten. Es gilt damit also $$P(A\cap B) = P(A) \cdot P_{A}(B)$$ und daraus folgt die Beziehung: $$P_{A}(B) = \dfrac{P(A \cap B)}{P(A)}$$
\end{multicols}
\end{merke*}
\begin{bsp}{Beispiel für eine bedingte Wahrscheinlichkeit}{}
Für einen Laplace-Würfel sind folgende Ereignisse gegeben:\\
\begin{enumerate}
    \item Ereignis A: Die geworfenen Augenzahl ist durch drei teilbar.
    \item Ereignis B: Die geworfene Augenzahl liegt zwischen 1 und 4.
\end{enumerate}
Nun soll die Wahrscheinlichkeit von A unter der Bedingung B bestimmt werden.\\
\emph{Es wird also die Wahrscheinlichkeit dafür gesucht, dass die gewürfelte Zahl durch drei teilbar ist unter der Bedingung, dass sie zwischen 1 und 4 liegt.}\\[0.5cm]
Für die Wahrscheinlichkeiten gilt damit folgendes: $$P(A) = \dfrac{2}{6} \hspace{0.2cm} \text{und} \hspace{0.2cm} P(B) = \dfrac{4}{6} $$
\scalebox{0.8}{
% allgemeines Layout des Baums
\tikzstyle{level 1}=[level distance=2.5cm, sibling distance=5cm]
\tikzstyle{level 2}=[level distance=2.5cm, sibling distance=4cm]
% definiert Knoten- und Endpunkte
% text width ändert die Boxbreite wobei 1em einem Zeichen entspricht
\tikzstyle{bag} = [circle, draw, text width=1em, inner sep=2pt, text centered]
\tikzstyle{end} = [circle, minimum width=4pt, fill, inner sep=0pt]
\begin{tikzpicture}[grow=down]
  \node[bag]{}
  child
  {
    node[bag] {$B$} % beschriftet Knoten 2 in erster Ebene mit X
    child
    {
      node (B) [bag] {$A$} % beschriftet Knoten 4 in zweiter Ebene mit D
      node [below of = B] {$P(B \cap A) = \dfrac{1}{6}$}
      edge from parent
      node[left]  {$P_{B}(A)$} % beschriftet Verbindung zu Knoten 4 (Ebene 2) mit p
} child {
      node[bag] {$\Bar{A}$} % beschriftet Knoten 3 in zweiter Ebene mit C
      edge from parent
      node[right]  {$P_{B}(\Bar{A})$} % beschriftet Verbindung zu Knoten 3 (Ebene 2) mit m
    }
    edge from parent
    node[left]  {$P(B)$} % beschriftet Verbindung zu Knoten 2 (Ebene 1) mit f
} child {
    node[bag] {$\Bar{B}$} % beschriftet Knoten 1 in erster Ebene mit A
    child
    {
      node[bag] {$A$} % beschriftet Knoten 2 in zweiter Ebene mit B
      edge from parent
      node[left]  {$P_{\Bar{B}}(A)$} % beschriftet Verbindung zu Knoten 2 (Ebene 2) mit k
} child {
      node[bag] {$\Bar{A}$} % beschriftet Knoten 1 in zweiter Ebene mit A
      edge from parent
      node[right]  {$P_{\Bar{B}}(A)$} % beschriftet Verbindung zu Knoten 1 (Ebene 2) mit i
    }
    edge from parent
    node[right]  {$P(\Bar{B})$} % beschriftet Verbindung zu Knoten 1 (Ebene 1) mit w
  };
\end{tikzpicture}}\\
Die Wahrscheinlichkeit für das Ereignis $A\cap B$ entspricht demjenigen Ereignis, dass sowohl B als auch A eintritt. Das entspricht der Menge die nur aus der Augenzahl 3 besteht und damit der Wahrscheinlichkeit $P(A\cap B) = \dfrac{1}{6}$. Aus dem Baumdiagramm folgt damit: $$P(A\cap B) = P(B) \cdot P_{B}(A)$$ und dadurch $$P_{B}(A) = \dfrac{P(A\cap B)}{P(B)} = \dfrac{\dfrac{1}{6}}{\dfrac{4}{6}}= \dfrac{1}{6} \cdot \dfrac{6}{4} = \dfrac{1}{4} $$.
\end{bsp}