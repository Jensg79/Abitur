\documentclass[a4paper,12pt]{article}
\usepackage{pgfplots}
\pgfplotsset{compat=1.15}
\usepackage{siunitx}
\sisetup{locale = DE} % Komma als Dezimaltrennzeichen

\begin{document}

\begin{center}
\begin{tikzpicture}
\begin{axis}[
    width=17.5cm,
    height=4.2cm,
    ybar,
    bar width=7.5pt,
    ymin=0, ymax=0.26,
    xmin=0, xmax=41,
    axis lines=left,
    axis line style={->, line width=0.9pt},
    tick align=outside,
    xtick={0,2,...,40},
    ytick={0,0.05,...,0.25},
    yticklabel style={
        /pgf/number format/fixed,
        /pgf/number format/precision=2,
        /pgf/number format/use comma
    },
    grid=major,
    grid style={gray!35, line width=0.6pt},
    major grid style={gray!35, line width=0.6pt},
    enlarge x limits=0.01,
    clip=false
]

% --- Parameter ---
\pgfmathsetmacro{\pp}{0.55}

% --- Helper: Binomial-PMF ---
% B(n,p;k) = binom(n,k) * p^k * (1-p)^(n-k)

% --- n = 10 (rot) ---
\addplot+[
    ybar,
    fill=red!80,
    draw=red!70!black,
    opacity=0.85
] coordinates {
% k = 0..10
\pgfmathsetmacro{\nn}{10}
\foreach \k in {0,...,10} {
    (\k, { (pgfmathbinom(\nn,\k)) * (\pp^(\k)) * ((1-\pp)^(\nn-\k)) })
}
};

% --- n = 25 (blau) ---
\addplot+[
    ybar,
    fill=cyan!70!blue,
    draw=cyan!60!black,
    opacity=0.85
] coordinates {
\pgfmathsetmacro{\nn}{25}
\foreach \k in {0,...,25} {
    (\k, { (pgfmathbinom(\nn,\k)) * (\pp^(\k)) * ((1-\pp)^(\nn-\k)) })
}
};

% --- n = 50 (gelb) ---
\addplot+[
    ybar,
    fill=yellow!85!orange,
    draw=yellow!60!black,
    opacity=0.85
] coordinates {
\pgfmathsetmacro{\nn}{50}
% nur bis 41 sichtbar (Achse endet bei 41)
\foreach \k in {0,...,41} {
    (\k, { (pgfmathbinom(\nn,\k)) * (\pp^(\k)) * ((1-\pp)^(\nn-\k)) })
}
};

% --- Beschriftungen wie im Bild (Positionen bewusst gesetzt) ---
\node[anchor=south west] at (rel axis cs:0.02,0.96) {$\mathrm{B}(n;\,0{,}55;\,k)$};

\node[anchor=west, text=red!80!black, font=\bfseries] at (axis cs:6.7,0.215) {n = 10};
\node[anchor=west, text=cyan!70!blue, font=\bfseries] at (axis cs:13.0,0.215) {n = 25};
\node[anchor=west, text=yellow!60!orange, font=\bfseries] at (axis cs:27.3,0.215) {n = 50};

\node[anchor=west] at (axis cs:18.6,0.155) {p = 0,55};

\node[anchor=west] at (axis cs:35.2,0.215) {n variabel};
\node[anchor=west] at (axis cs:35.2,0.165) {p konstant};

% x-Achsenlabel "k" am Pfeilende
\node[anchor=west] at (rel axis cs:1.01,0.10) {$k$};

\end{axis}
\end{tikzpicture}
\end{center}

\end{document}